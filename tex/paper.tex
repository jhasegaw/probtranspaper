\documentclass[11pt,draftcls,onecolumn,peerreview]{IEEEtran}
\bibliographystyle{plain}
\usepackage{graphicx}
\usepackage{amssymb}
\usepackage{amsmath}
\usepackage[bbgreekl]{mathbbol}
\DeclareMathOperator*{\argmax}{argmax}
\DeclareMathOperator*{\argmin}{argmin}
\usepackage[affil-it]{authblk}
\usepackage{paralist}
\usepackage{url}
\usepackage{tabularx}
\usepackage{color}
\usepackage{tikz}
\usepackage{tipa}\newcommand{\ipa}[1]{\textipa{#1}}
\usepackage{stackrel}
\usetikzlibrary{positioning,shadows,arrows,shapes,calc}
\setlength{\textwidth}{6.5in}
\setlength{\textheight}{9in}
\setlength{\oddsidemargin}{0in}
\setlength{\topmargin}{0in}

\title{ASR for Under-Resourced Languages from Probabilistic Transcription}

\author{Mark Hasegawa-Johnson$^1$,~\IEEEmembership{Senior~Member,~IEEE}
  Adrian KC Lee$^2$,~\IEEEmembership{Member,~IEEE}
  Ed Lalor$^3$,
  Preethi Jyothi$^1$,~\IEEEmembership{Member,~IEEE}
  Daniel McCloy$^2$,
  Majid Mirbagheri$^2$,
  Giovanni di Liberto$^3$,
  Amit Das$^1$,
  Bradley Ekin$^2$,~\IEEEmembership{Student Member,~IEEE}
  Chunxi Liu$^4$,
  Vimal Manohar$^4$,
  Hao Tang$^5$,
  Nancy Chen$^6$,~\IEEEmembership{Senior~Member,~IEEE}
  Paul Hager$^7$,
  Tyler Kekona$^2$,
  and Rose Sloan$^8$}
\affil{1. University of Illinois, 2. University of Washington,
  3. Trinity College, Dublin, 4. Johns Hopkins University, 5. Toyota
  Technological Institute Chicago, 6. Institute for Infocomm Research,
  7. MIT, 8. Yale University}

\markboth{Draft, for review only}
{Hasegawa-Johnson \MakeLowercase{\textit{et al.}}: ASR for Under-Resourced Languages from Probabilistic Transcription}

\begin{document}
\maketitle

\begin{abstract}
In many under-resourced languages it is possible to find text and it is possible to find speech, but transcribed speech suitable for training automatic speech recognition (ASR) is unavailable.  In the absence of native transcript, this paper proposes the use of a probabilistic transcript: a probability mass function over possible phonetic transcripts of the waveform.  Three sources of probabilistic transcript are demonstrated.  First, self-training is a well-established semi-supervised learning technique, in which a cross-lingual ASR first labels unlabeled speech, and is then adapted using the same labels.  Second, mismatched crowdsourcing is a recent technique in which non-speakers of the language are asked to write what they hear, and their nonsense transcripts are decoded using noisy channel models of second-language speech perception.  Third, EEG distribution coding is a new technique in which non-speakers of the language listen to it, and their electrocortical response signals are interpreted to indicate probabilities.  ASR was trained in four languages with no transcribed training speech.  Adaptation using mismatched crowdsourcing significantly outperformed self-training, and both significantly outperformed a cross-lingual baseline.  EEG distribution coding and text-derived phone language models were both shown to improve the quality of probabilistic transcripts derived from mismatched crowdsourcing.

\end{abstract}

\begin{IEEEkeywords}
Automatic speech recognition, Under-resourced languages, Crowdsourcing, EEG
\end{IEEEkeywords}

\ifCLASSOPTIONpeerreview
\begin{center} \bfseries EDICS Category: SPE-RECO \end{center}
\fi
% For peerreview papers, this IEEEtran command inserts a page break and
% creates the second title. It will be ignored for other modes.
\IEEEpeerreviewmaketitle

%%%%%%%%%%%%%%%%%%%%%%%%%%%%%%%%%%%%%%%%%%%
\section{Introduction}

\IEEEPARstart{A}{utomatic} speech recognition (ASR) has the potential to provide
database access, simultaneous translation, and text/voice messaging
services to anybody, in any language, dramatically reducing linguistic
barriers to economic success.  To date, ASR has failed to achieve its
potential, because successful ASR requires very large labeled
corpora;
%. Current methods require about 1000 hours of transcribed
%speech per language, transcribed at a cost of about 6000 hours of
%human labor;
the human transcribers must be computer-literate, and
they must be native speakers of the language being transcribed.
%In
%many languages, recruiting dozens of computer-literate
%native speakers is impractical.
{\color{blue} Large corpora are beyond the resources of most
  under-resourced language communities; we have found that
  transcribing even one hour of speech may be beyond the reach of
  communities that lack large-scale government funding.  In order to
  create the databases reported in this paper, for example, we sought
  paid native transcribers, at a competitive wage, for the 68
  languages in which we have untranscribed audio data.  We found
  transcribers willing to work in only eleven of those languages, of
  which only seven finished the task.}

Instead of recruiting native transcribers in search of a perfect
reference transcript, this paper proposes the use of probabilistic
transcripts.  A probabilistic transcript is a probability mass
function, $\rho_\Phi(\phi)$, specifying, as a real number between $0$ and
$1$, the probability that any particular phonetic transcript $\phi$
is the correct transcript of the utterance.  Prior to this work,
machine learning has almost always assumed that the training dataset
contains either deterministic transcripts
($\rho_{DT}(\phi)\in\left\{0,1\right\}$, commonly called ``supervised
training'') or completely untranscribed utterances (commonly called
``unsupervised training,'' in which case we assume that $\rho_{LM}(\phi)$
is given by some {\em a priori} language model).  This article
proposes that, even in the absence of a deterministic transcript,
there may be auxiliary sources of information that can be compiled to
create a probabilistic transcript with entropy lower than that
of the language model, and that machine learning methods applied to
the probabilistic transcript are able to make use of its reduced
entropy in order to learn a better ASR.  In particular,
this paper considers three useful auxiliary sources of information:
\begin{enumerate}
\item SELF-TRAINING: ASR pre-trained in other languages is used to
  transcribe unlabeled training data in the target language.
\item MISMATCHED CROWDSOURCING: Human crowd workers who don't speak
  the target language are asked to transcribe it as if it were a
  sequence of nonsense syllables.
\item EEG DISTRIBUTION CODING: Humans who do not speak the target
  language are asked to listen to its extracted syllables, and their
  EEG responses are interpreted as a probability mass function over
  possible phonetic transcripts.
\end{enumerate}



%%%%%%%%%%%%%%%%%%%%%%%%%%%%%%%%%%%%%%%%%%%%%%%%%%%%%%%%%%%%%%
\section{Background}

Suppose we require that, in
order to develop speech technology, it is necessary first to have (1)
some amount of recorded speech audio, and (2) some amount of text
written in the target language.  These two requirements can be met by
at least several hundred languages: speech audio can be recorded
from podcasts or radio broadcasts,
and text can be acquired from Wikipedia, Bibles, and textbooks.
Recorded speech is, however, not usually transcribed; and the
requirement of native language transcription is beyond the economic
capabilities of many minority-language communities.

\subsection{ASR in Under-Resourced Languages}

Krauwer~\cite{Krauwer2003} defined an under-resourced language to be
one that lacks one or more of: stable orthography, significant
presence on the internet, linguistic expertise, monolingual tagged
corpora, bilingual electronic dictionaries, transcribed speech,
pronunciation dictionaries, or other similar electronic resources.
Berment~\cite{Berment2004} defined a rubric for tabulating the
resources avaialble in any given language, and proposed that a
language should be called ``under-resourced'' if it scored lower than
10.0/20.0 on the proposed rubric.  By these standards, technology
methods for under-resourced languages are most often demonstrated on
languages that are not really under-resourced: for example, ASR may be
trained without transcribed speech, but the quality of the resulting
ASR can only be scientifically proven by measuring its phone error
rate (PER) or word error rate (WER) using transcribed speech.  The
intention, in most cases, is to create methods that can later be
ported to languages that truly lack resources.

The International Phonetic Alphabet (IPA~\cite{ipa1993}) is a set of
phoneme symbols defined by the principle that, if there exists a
language that distinguishes two phonemes, then they should have
distinguishable symbols. ASR in a new language can be rapidly deployed
using acoustic models trained to represent every distinct symbol in
the IPA~\cite{Schultz2001}.  Equality between the IPA symbols for
phones in two languages, however, does not mean that the corresponding
acoustic category boundaries coincide, even between dialects of the
same language: a monolingual Gaussian mixture model (GMM) trained on
five hours of Levantine Arabic can be improved by adding ten hours of
Standard Arabic data, but only if the log likelihood of cross-dialect
data is scaled by 0.02~\cite{Huang2012}.  Stronger effects of
cross-language transfer are available only by using structured
transfer learning methods, including neural networks (NN) and subspace
Gaussian mixture models (SGMM).

NN transfer learning can be categorized as tandem, bottleneck,
pre-training, phone mapping, and multi-softmax methods.  In a tandem
system, outputs of the NN are Gaussianized, and used as features whose
likelihood is computed with a GMM~\cite{Hermansky2000}; in a
bottleneck system, features are extracted from a hidden layer rather
than the output layer. Both tandem~\cite{Stolcke2006} and
bottleneck~\cite{Vesely2012} features trained on other languages can
be combined with GMMs trained on the target language in order to
improve word error rate (WER).

A hybrid ASR is a system in which the NN terminates in a softmax
layer, whose outputs are interpreted as phone~\cite{Morgan95} or
senone~\cite{Dahl2012} probabilities.  Knowledge of the target
language phone inventory is necessary to train a hybrid ASR, but it is
possible to reduce WER by first pre-training the NN hidden layers with
multilingual data~\cite{Huang2013,Swietojanski2012}.  A hybrid ASR can be
constructed using very little in-language speech data by adding a
single phone-mapping layer to the output of the multilingual NN; the
phone mapping layer can be trained using a small amount of in-language
speech data~\cite{Sim2008}, even if context-dependent senones are
mapped instead of phones~\cite{Do2012}.  A multi-softmax system
integrates phone mapping into the original training procedure, by
training a network with several different language-dependent softmax
layers, each of which is the linear transform of a multilingual shared
hidden layer.  Multi-softmax systems have reduced WER in
tandem~\cite{Scanzio2008}, bottleneck~\cite{Vesely2012}, and
hybrid~\cite{Huang2013} ASR.

SGMM transfer learning uses language-dependent GMMs, each of which is
the linear interpolation of language-independent mean and variance
vectors.  SGMM can be combined with other methods for further
improvement, e.g., 16\% relative WER reduction was achieved in a Tamil
ASR by combining SGMM with an acoustic data normalization
technique~\cite{Mohan2014}, and further reductions were obtained in
Afrikaans by using bottleneck features in an SGMM~\cite{Imseng2014}.

Self-training is a class of semi-supervised learning techniques in
which ASR is first trained on labeled corpora in other languages, then
used to label data in the target language.  Self-labeled data in the
target language is then used to train or adapt the
ASR~\cite{Loof2009,Cetin2008}.  Self-training is most useful when the
in-language training data are first filtered, to exclude frames with
confidence below a threshold.  The posterior probability computed from
the ASR lattice is a useful confidence score~\cite{vesely2013-semi},
but it is also possible to learn an improved confidence score by
combining multiple sources of information~\cite{Vu2011b}.

Under-resourced languages often lack any pronunciation dictionary.  It
is possible to train a stable grapheme-to-phoneme transducer using a
dictionary with 15,000 entries, and in some languages a dictionary of
this size can be mined from sources such as
Wiktionary~\cite{Schlippe2014}.  In languages without any dictionary
of this size, it may be possible to approximate pronunciation by
treating each orthographic character as an acoustic model (a
grapheme)~\cite{Kanthak2002,Charoenpornsawat06,Gizaw2008,Le2009}.
Even an ambiguous G2P can often be disambiguated by the use of
context-dependent graphemic models~\cite{Kanthak2002}; if the number
of trigraphemes gets too large, acoustic models can be interpolated
within an eigentrigrapheme space~\cite{Ko2014}.  Optimal WER in
Standard Arabic was achieved by using phoneme-based pronunciations for
the most frequent 500 words, and grapheme-based pronunciations for all
less frequent words~\cite{Elmahdy2012}.  In Amharic, optimal WER was
achieved using a morpheme-based language model, combined with a hybrid
acoustic model space including both triphones and context-dependent
sylabic units~\cite{Tachbelie2014}.  In Hindi, optimal WER was
achieved using a one-to-one character-based grapheme-to-phoneme (G2P)
transducer (essentially a grapheme-based acoustic model), modified by
a very small set (3) of surface phonological
rules~\cite{Jyothi2015interspeech_hindi}.  The three rules were
proposed based on phonological descriptions of Hindi, then applied or
discarded in response to application probabilities learned using a
very small (200-word) pronunciation dictionary.


\subsection{Mismatched Crowdsourcing}
\label{sec:bgmc}

In~\cite{JHJ15a}, a methodology was proposed that bypasses the need
for native language transcription: mismatched crowdsourcing sends target
language speech to crowd-worker transcribers who have no
knowledge of the target language, then uses explicit mathematical
models of second language phonetic perception to recover an equivalent
phonetic transcription (Fig.~\ref{fig:h2e_eg2}).  Majority voting is
re-cast, in this paradigm, as a form of error-correcting code
(redundancy coding), which effectively increases the capacity of the
noisy channel; interpretation as a noisy channel permits us to explore
more effective and efficient forms of error-correcting codes.

\begin{figure}[b!]\setlength{\textfloatsep}{3mm}
\begin{center}
  \tikzstyle{pre2}=[<-,>=stealth',thick, draw=black]
  \tikzstyle{post2}=[->,>=stealth',thick, draw=black]
  \begin{tikzpicture}[
      scale=\mytikzscale,
      boxed/.style={rectangle,thick, draw=black, text=black, rounded corners=1mm, text centered, text width=2.5cm},
      open/.style={text=black, text centered, text width=2.75cm},
      every node/.style={transform shape}      
    ]
    \node[open] (n0) at (-4.5,1) {$w=$Utterance-language words\\\vspace*{0.1in}$<$$\vcenter{\hbox{\includegraphics[width=1in]{../figs/vaakpahachaan.png}}}$$>$};
    \node[boxed] (n1) at (-0.75,1) {Pronunciation FST,\\$\rho(\phi|w)$} 
    edge[pre2](n0);
    \node[boxed] (n2) at (2.5,1) {Misperception FST,\\$\rho(\psi|\phi)$}
    edge[pre2] (n1);
    \node[boxed] (n3) at (5.75,1) {Phoneme-to-grapheme FST, $\rho(\lambda|\psi)$}
    edge[pre2] (n2);
    \node[open] (n4) at (9,1) {$\lambda$: Annotation-language orthography\\$<$vak paychan$>$} 
    edge[pre2](n3);
    \node[open] (n5) at (1,-1.5) {$\phi=$Utterance phones\\\ipa{[va:k p@H@tSa:n]}};
    \node[open] (n6) at (4,-1.5) {$\psi=$Perceived phones\\\ipa{[vAk p\textsuperscript{h}eItSAn]}}; 
  \end{tikzpicture}\\
\end{center}
\setlength{\abovecaptionskip}{0pt}
\caption{Mismatched Crowdsourcing: crowd workers on the web are asked
  to transcribe speech in a language they do not know.  Annotation
  mistakes are modeled by a finite state transducer (FST) model of
  utterance-language pronunciation variability (reduction and
  coarticulation), composed with an FST model of non-native speech
  misperception (mapping utterance-language phones to
  annotation-language phones), composed with an inverted
  grapheme-to-phoneme (G2P) transducer.}
\label{fig:h2e_eg2}
\end{figure}

Assume that cross-language phone misperception is a finite-memory
process, and can therefore be modeled by a finite state transducer
(FST).  The complete sequence of representations from utterance
language to annotation language can therefore be modeled as a noisy
channel represented by the composition of up to three consecutive FSTs
(Fig.~\ref{fig:h2e_eg2}): a pronunciation model, a misperception
model, and an inverted grapheme-to-phoneme (G2P) transducer.  The
pronunciation model is an FST representing processes that distort the
canonical phoneme string during speech production, including processes
of reduction and coarticulation.  The misperception model represents
the mapping of the uttered phone string (in symbols matching the
phone set of the spoken language) to the perceived phone string (in
symbols matching the phone set of the annotation language).  Finally,
the transcriber maps heard phones to nonsense words in the annotation
language; the mapping from phones to orthography is an inverted G2P.


\subsection{Electrophysiology of Speech Perception}
  
EEG distribution coding is a proposed new method that interprets the
pre-categorical electrical evoked potentials of untrained listeners
(measured by an electro-encephalograph or EEG) as a posterior
probability distribution over the phone set of the utterance language
(Fig.~\ref{fig:eeg_paradigm}).  Transcribers, in this scenario, first
listen to speech in their native language, and their EEG responses are
recorded.  From their responses to English speech, an English-language
EEG phone recognizer is trained, using methods based
on~\cite{Liberto15}.  In order to estimate the misperception
probabilities $\rho(\psi|\phi)$, then, non-native syllables are played
to the same listeners.  For each non-native phone $\phi$, the
classifier outputs are interpreted as an estimate of $\rho(\psi|\phi)$
for all $\phi\in\mathbb{\Phi}$, the native phone inventory.

\begin{figure}
  \centerline{\includegraphics[width=4in]{../figs/diliberto_paradigm.png}}
  \caption{EEG responses are recorded while listeners hear speech in
    their native language.  For each listener, a bank of distinctive
    feature classifiers are trained.  The same listeners then hear
    speech in a non-native language, and the same classifiers are
    applied, estimating a listener-language transcription of the
    non-native speech.}
  \label{fig:eeg_paradigm}
\end{figure}



%%%%%%%%%%%%%%%%%%%%%%%%%%%%%%%%%%%%%%%%%%%%%%%%
\section{Algorithms that Induce a Probabilistic Transcription}

Three different experimental sources were tested for the creation of a
PT.  Self-training is now well-established in the field of
under-resourced ASR; we adopted the algorithm of Vesely, Hannemann and
Burget~\cite{vesely2013-semi}.  Mismatched crowdsourcing used original
annotations collected using our own previously published
methods~\cite{JHJ15b}.  EEG was not used independently here, but
rather, was used to learn a misperception model applicable to the
interpretation of mismatched crowdsourcing.

\subsection{Self Training}
\label{sec:selftraining}

The method used for semi-supervised training is a modification of the
self-training approach described in \cite{vesely2013-semi}. In this
method, a multilingual DNN-HMM speech recognizer, trained languages
other than the target language, is used to decode unlabelled audio in
the target language.  As shown in Fig.~\ref{fig:fig_hager}, decoding
results in a posterior probability $\pi(\phi_m^\ell|x_t^\ell)$ for
each frame $x_t^\ell$ of audio in the target language.

\begin{figure}
  \centerline{\includegraphics[width=5in]{../figs/fig_hager.png}}
  \caption{The self-training method of~\cite{vesely2013-semi} includes
    a labeling phase and a learning phase.  (a) Labeling phase: an ASR
    trained on other languages (here Cantonese) is used to compute
    posterior phone probabilities $\pi(\phi_t^\ell|x^\ell)$ in the
    test language (here Mandarin). (b) Learning phase: posterior phone
    probabilities are used as targets for DNN re-training.}
  \label{fig:fig_hager}
\end{figure}

We empirically found it better to use the posteriors as soft-targets
in frame cross-entropy training (Fig.~\ref{fig:fig_hager}). This is
different from the approach in \cite{vesely2013-semi}, which uses the
best path alignment as the target. Additionally, we scaled the amount
of transcribed data by 2 to create a good balance between transcribed
and untranscribed data as suggested in the original work.

The results on using this DNN are shown in Table \ref{tab:ptresult}. Although
semi-supervised training improves PER performance over multilingual DNN, it
still falls short of adaptation to probabilistic transcriptions (described in
Section \ref{sec:adaptation}). This is in spite of the untranscribed audio data
being several times larger than the probabilistic transcription data. Thus, we
show that mismatch transcripts can be more effective than ASR transcription for
training acoustic models.


\subsection{Mismatched Crowdsourcing}
\label{sec:MC}

\begin{figure}[b!]\setlength{\textfloatsep}{3mm}
\begin{center}
  \tikzstyle{pre}=[<-,shorten <=3pt,>=stealth',thick, draw=black]
  \tikzstyle{post}=[->,shorten >=3pt,>=stealth',thick, draw=black]
  \begin{tikzpicture}[
      scale=\mytikzscale,
      boxed/.style={rectangle,thick, draw=black, text=black, rounded corners=1mm, text centered, text width=5cm},
      state/.style={circle,thick, draw=black, text=black, text width=0.25cm},
      open/.style={text=black, text centered},
      every node/.style={transform shape}
    ]
    \node[open] (n0) at (1,0) {\begin{tabular}{c}$T$=Mismatched\\Transcripts\\\hline trabiza\\ta peesome\\ta pisha\\chah peesh um\\shapisha\\sabeesham\\chapiser\\some pizza\end{tabular}};
    \node[boxed] (n1) at (6,0) {\begin{tabular}{c}$\rho(\lambda|T)$\\\hline\vspace{3cm}\end{tabular}} edge[pre] (n0);
    \node[state] (g0) at (4,-0.25) {};
    \node[state] (g1) at (6,-0.25) {};
    \node (g2) at (8,-0.25) {\ldots};
    \draw[post] (g0) -- (4.25,0.75) -- (5.75,0.75) -- (g1);
    \node at (5,1) {$<$t$>$$/0.4$};
    \draw[post] (g0) -- (g1);
    \node at (5,0) {$<$ch$>$$/0.4$};
    \draw[post] (g0) -- (4.25,-1.25) -- (5.75,-1.25) -- (g1);
    \node at (5,-1) {$<$s$>$$/0.2$};
    \draw[post] (g1) -- (g2);
    \node at (7,0) {$<$a$>$$/1.0$};
    \node[boxed] (n2) at (12,0) {\begin{tabular}{c}$\rho(\phi|T)$\\\hline\vspace*{3cm}\end{tabular}} edge[pre] (n1);
    \node[state] (g10) at (10,-0.25) {};
    \node[state] (g11) at (12,-0.25) {};
    \node (g12) at (14,-0.25) {\ldots};
    \draw[post] (g10) -- (10.25,0.75) -- (11.75,0.75) -- (g11);
    \node at (11,1) {\ipa{[t]}$/0.4$};
    \draw[post] (g10) -- (g11);
    \node at (11,0) {\ipa{[tS]}$/0.4$};
    \draw[post] (g10) -- (10.25,-1.25) -- (11.75,-1.25) -- (g11);
    \node at (11,-1) {\ipa{[s]}$/0.2$};
    \draw[post] (g11) -- (g12);
    \node at (13,0) {\ipa{[A]}$/1.0$};
  \end{tikzpicture}\\
\end{center}
\setlength{\abovecaptionskip}{0pt}
\caption{Probabilistic transcription from mismatched crowdsourcing:
  Transcripts $T$ are filtered to remove outliers, and merged to
  create a confusion network over orthographic symbols,
  $\rho(\lambda|T)$, from which the probabilistic transcript
  $\rho(\phi|T)$ is inferred. Example shown: Swahili speech,
  English-speaking transcribers.  Symbols in $<$$>$ are graphemes,
  symbols in $[]$ are phones, numbers are probabilities.}
\label{fig:mcmethods}
\end{figure}

The second set of PTs was computed by sending audio in the target
language to non-speakers of the target language, and asking them to
write what they hear.  It would be preferable to recruit transcribers
who speak a language with predictable orthography, but since
transcribers in those languages were more expensive, this
experiment instead recruited transcribers who speak American English.
Denote using $T$ the set of mismatched
transcripts produced by these English-speaking crowd workers,
which we wish to interpret as a pmf over
target-language phone sequences, $\rho(\phi|T)$.  As an intermediate
step, prior work~\cite{JHJ15b} developed techniques
to merge texts into a confusion network
$\rho(\lambda|T)$ over representative transcripts in the
annotation-language orthography (Fig.~\ref{fig:mcmethods}).
%Formation of
%$\rho(\lambda|T)$ involves data filtering to remove outliers (based on
%pair-wise string edit distance among transcripts), expansion of the
%orthography to an alphabet that includes
%single-character symbols for digraphs and sequences commonly used to represent
%single phonemes in English orthography ($<$ai, ay, ee, oo, ou, aw, ow,
%bh, ch, dh, gh, jh, kh, ph, sh, th, wh, zh, ck$>$, and any vowel followed by
%a word-final silent $<$e$>$), and a weighted
%voting scheme in which the weight of each transcript is proportional
%to the frequency with which it matches the other transcripts.

Once transcripts have
been aligned and filtered to create the orthographic confusion network
$\rho(\lambda|T)$, they are then translated into a distribution over
phone transcripts according to:
\begin{align}
  \rho(\phi|T) 
%  &=\sum_{\lambda} \rho(\phi|\lambda,T) \rho(\lambda|T) \notag \\
  &\approx \max_{\lambda}  \rho(\phi|\lambda) \rho(\lambda|T) \notag \\
  &= \max_{\lambda}  \left(\frac{\rho(\lambda|\phi)}{\rho(\lambda)}
  \rho(\phi)\right) \rho(\lambda|T) 
\label{eq:PT}
\end{align}
The terms other than $\rho(\lambda|T)$ in Equation~(\ref{eq:PT}) are
estimated as follows.  $\rho(\lambda)$ is modeled using a unigram
prior over the letter sequences in $\lambda$.  $\rho(\phi)$ is modeled
using either a cross-lingual phone unigram, a
language-constrained cross-lingual unigram (the cross-lingual
unigram, constrained to take values from the phone set of the target
language), or a language-specific phone bigram
$\rho(\phi)=\prod_{m=1}^M \rho(\phi_m|\phi_{m-1})$.
Sec.~\ref{sec:trainwithlm} describes an algorithm for training the
phone bigram without using proscribed test-language resources;
Sec.~\ref{sec:data} lists the PT accuracies achieved using each of
these three approaches.
$\rho(\lambda|\phi)$ is called the misperception G2P, as it maps to
graphemes in the annotation language, $\lambda$, fsrom phones in the
utterance language, $\phi$.  Section~\ref{sec:eegchanmod} describes
methods that decompose $\rho(\lambda|\phi)$ into separate
misperception and G2P transducers, but it can also be trained directly
using
representative transcripts $\lambda$ (and their
corresponding native transcripts) for speech {\em in languages other
  than the target language}.
The model learned in this way is essentially
a machine translation model, which translates between graphemes in
the annotation language ($\lambda$) to phonemes in any possible
utterance language ($\phi$).
We assume that misperceptions depend more
heavily on the annotation language than on the utterance language, and
that therefore a model $\rho(\lambda|\phi)$ trained using a universal
phone set for $\phi$ is also a good model of $\rho(\lambda|\phi)$ for
the target language. Note that, while this assumption is not entirely
accurate, it is necessitated by the requirement that no native
transcripts in the target language can be used in building any part
of our system.


\subsection{Channel Model from EEG}

Mismatched crowdsourcing depends on a misperception G2P, which can be
trained from doubly-transcribed data in languages other than the test
language, as described in the previous section.  With a small amount
of transcribed data in the utterance language, however, it is possible
to estimate the misperception G2P using electrocortical measurements
of non-native speech perception, as follows.

The misperception G2P can be decomposed into two separate transducers,
a misperception transducer $\rho(\psi|\phi)$, and an
annotation-language G2P $\rho(\lambda|\psi)$:
\begin{equation}
  \Pr(\lambda|\psi)=\sum_{\psi}\Pr(\lambda|\psi)\Pr(\psi|\phi)
\end{equation}
where $\phi$ is a phone string in the utterance language, $\psi$ is a
phone string in the annotation language, and $\lambda$ is an
orthographic string in the annotation language.  $\rho(\lambda|\psi)$
is an inverted G2P in the annotation language, e.g., trained on the
CMU dictionary of American English pronunciations~\cite{Lenzo1995}.
$\rho(\psi|\phi)$ is the mismatch transducer, specifying the
probability that a phone string $\phi$ in the utterance language will
be mis-heard as the annotation-language phone string $\psi$.

Distinctive features were proposed to characterize the perceptual and
phonological natural classes of phonemes~\cite{Jakobson52}, therefore
the distance between the distinctive feature representations of two
phonemes is an indirect predictor of their confusion probability.  Let
$f_k(\phi)$ be the $k^{\textrm{th}}$ feature of phoneme $\phi$, and
let $\delta(\cdot)\in\left\{0,1\right\}$ be the unit indicator
function.  The assumption that every distinctive feature shared by
phonemes $\phi$ and $\psi$ independently increases their confusion
probability can be expressed as
\begin{equation}
  \rho(\psi|\phi)\propto \exp\left(-\sum_{k=1}^K
  w_k\delta\left(f_k(\psi)\ne f_k(\phi)\right)\right)
  \label{eq:dfdist}
\end{equation}
The weights $w_k$ probably depend on the listener, and on the
identities of both speaker language and listener language, but data to
train such a rich model do not exist; a reasonable approximate model
can be learned by assuming that $w_k$ depend only on the language of
the listener.

Distinctive features were defined according to the PHOIBLE inventory.
The feature weights, $w_k$, were set differently in different
experiments.  In some experiments, they were set to be uniform.
Better accuracy was obtained, however, by setting them on the basis of
perceptual similarity as measured using EEG.

Denote as $y(\psi)$ the set of EEG signals recorded when a listener
hears audio corresponding to phoneme $\psi$ in the annotation
language, and suppose that $g_k(y(\psi))$ is the output of a binary
classifier of EEG signals trained to label the $k^{\textrm{th}}$
distinctive feature of phoneme $\psi$, as in~\cite{Liberto15}.  Then
the weights in Eq.~\ref{eq:dfdist} can be estimated as
\begin{equation}
  w_k = -\ln\Pr\left\{g_k(y(\psi))\ne f_k(\psi)\right\}
  \label{eq:eegdist}
\end{equation}



%%%%%%%%%%%%%%%%%%%%%%%%%%%%%%%%%%%%%%%%%%%%%%%%
\section{Algorithms for Training ASR Using Probabilistic Transcription}

An ASR is a parameterized pmf,
$\pi(x,s|\phi,\theta)$, specifying the dependence of
acoustic features, $x$, and senones, $s$, on the phone transcript
$\phi$ and the parameter vector $\theta$, where the notation
$\pi(\cdot)$ denotes a pmf dependent on ASR parameters.  
Assume a hidden Markov model (HMM), therefore
\[
\pi(x,s|\phi,\theta)=\prod_{\ell=1}^L \prod_{t=1}^T
\pi(s_t^\ell|s_{t-1}^\ell,\phi^\ell,\theta)\pi(x_t^\ell|s_t^\ell,\theta)
\]

\subsection{Maximum Likelihood Training}

Consider two observation-conditional sequence distributions
$\pi(s,\phi|x,\theta)$ and $\pi(s,\phi|x,\theta')$, with parameter
vectors $\theta$ and $\theta'$ respectively.  The cross-entropy
between these distributions is~\cite{Dempster77}:
\begin{align}
  H\left(\theta\Vert\theta'\right) &=
  \sum_{s,\phi} \pi(s,\phi|x,\theta)
  \ln \pi(s,\phi|x,\theta')\\
  &=   \sum_{s,\phi} \pi(s,\phi|x,\theta)
  \left(\ln \pi(s,\phi,x|\theta')-\ln \pi(x|\theta')\right)\\
  &=  Q\left(\theta,\theta'\right)-{\mathcal L}\left(\theta'\right)
  \label{eq:crossentropy}
\end{align}
where the data log likelihood, ${\mathcal L}\left(\theta'\right)$, and
the expectation maximization (EM) quality function,
$Q\left(\theta,\theta'\right)$, are defined by
\begin{align}
  {\mathcal L}\left(\theta'\right) &= \ln \pi(x|\theta')
  \label{eq:loglikelihood}\\
  Q\left(\theta,\theta'\right)
  &=
  \sum_{s,\phi} \pi(s,\phi|x,\theta)\ln \pi(s,\phi,x|\theta')
   \label{eq:Qfunction}
\end{align}
The Kullback-Leibler divergence between $\pi(s,\phi|x,\theta)$ and
$\pi(s,\phi|x,\theta')$ is $D\left(\theta\Vert\theta'\right)=
H\left(\theta\Vert\theta'\right)-H\left(\theta\Vert\theta\right)$.
Since $D\left(\theta\Vert\theta'\right)\ge 0$~\cite{Shannon49},
\begin{equation}
  {\mathcal L}\left(\theta'\right)-{\mathcal L}\left(\theta\right)\ge
  Q\left(\theta,\theta'\right)-
  Q\left(\theta,\theta\right)
  \label{eq:LgeQ}
\end{equation}
Given any initial parameter vector $\theta_n$, the expectation
maximization (EM) algorithm finds $\theta_{n+1}=\argmax
Q(\theta_n,\theta')$, thereby maximizing the minimum increment in
${\mathcal L}(\theta)$.  For GMM-HMMs, the quality function
$Q\left(\theta,\theta'\right)$ is convex and can be analytically
maximized; for DNN-HMMs it is non-convex, but can be maximized using
gradient ascent.
%% above paragraph is the first appearance of "DNN", which has not yet
%% been defined (though "NN" has).

The probability $\pi(x,s,\phi|\theta)$ is computed by composing the
following three weighted FSTs:
\begin{align}
  \mathbf{PT}&:\phi^\ell\rightarrow\phi^\ell/ \rho(\phi^\ell)\\
  \mathbf{HC}&:\phi^\ell\rightarrow s^\ell/ \pi(s^\ell|\pi^\ell,\theta)\\
  \mathbf{AM}&:s^\ell\rightarrow s^\ell/ \pi(x^\ell|s^\ell,\phi^\ell,\theta)
\end{align}
where the notation has the following meaning.  The probabilistic
transcription, $\mathbf{PT}$, is an FST that maps any phone string
$\phi^\ell\in\mathbb{\Phi}^*$ to itself.  This mapping is
deterministic and reflexive, but comes with a path cost determined by
the transcription probability $\rho(\phi^\ell)$, as exemplified in
Fig.~\ref{fig:pt}.  The HMM-expansion transducer, $\mathbf{HC}$, maps
any phone sequence $\phi^\ell$ to a state sequence $s^\ell$.  This FST
is the composition of the $\mathbf{H}$ and $\mathbf{C}$ transducers
defined by~\cite{Mohri2002}.  This mapping is non-deterministic, and
the path cost is determined by the HMM transition weights distribution
$a_{ij}=\pi(s_t^\ell =j|s_{t-1}^\ell =i,\phi^\ell,\theta)$:
\begin{equation}
  \pi(s^\ell|\phi,\theta)=\prod_{\ell=1}^L\prod_{t=1}^T
  a_{s_{t-1}^\ell s_t^\ell}
\end{equation}
The acoustic modeling transducer $\mathbf{AM}$ maps any state sequence
to itself.  This mapping is deterministic and reflexive, but comes
with a path cost determined by the acoustic modeling probability
\begin{equation}
  \pi(x^\ell|s^\ell,\phi^\ell,\theta)=\prod_{\ell=1}^L\prod_{t=1}^T
  \pi(x_t^\ell|s_t^\ell,\theta^\ell)
\end{equation}
The joint probability $\pi(\phi^\ell,s^\ell,x^\ell|\theta)$ is
computed by composing the FSTs, then finding the total cost of the
path through
$\left(\mathbf{AM}\circ\mathbf{HC}\circ\mathbf{PT}\right)$ with input
string $\phi^\ell$ and output string $s^\ell$.  The posterior
probability $\pi(\phi^\ell,s^\ell|x^\ell,\theta)$ is computed by
pushing the composed FST, then finding the total cost of the path
through
$\textrm{push}\left(\mathbf{AM}\circ\mathbf{HC}\circ\mathbf{PT}\right)$.

%% single-sentence paragraph (consider merging into above or below
%% paragraphs)
The parameter vector $\theta$ includes the HMM transition
probabilities, $a_{ij}=\pi(s_t^\ell =j|s_{t-1}^\ell =i,\phi,\theta)$,
and the parameters of the acoustic model
$b_j(x_t^\ell)=\pi(x_t^\ell|s_t^\ell=j,\theta)$.

Computing the analytical maximum or gradient of
$Q\left(\theta,\theta'\right)$ requires summation over all possible
state alignments $s\sim S$.  The summation can be performed
efficiently using the Baum-Welch algorithm, but experimental tests
reported in this paper did not do so, for reasons described in the
next subsection.

\subsection{Segmental K-Means Training}

The EM quality function, $Q(\theta,\theta')$,
has properties that make it undesirable as an optimizer for ${\mathcal{L}}$.
Suppose, as often happens, that there is a poor phone
sequence, $\phi^p$, that is highly unlikely given the correct
parameter vector $\theta^*$, meaning that $\pi(x,s,\phi^p|\theta^*)$
is very low.  Suppose that the initial parameter vector, $\theta$, is
less discriminative, so that
$\pi(x,s,\phi^p|\theta)>\pi(x,s,\phi^p|\theta^*)$.
Indeed, the best speech recognizer is a parameter vector $\theta^*$
that completely rules out poor transcripts, setting
$\pi(x,s,\phi^p|\theta^*)=0$; but in this case
$Q(\theta,\theta^*)=-\infty$.
It is therefore not possible for the EM algorithm to start with
parameters $\theta$ that allow $\phi^p$, and to find parameters $\theta^*$
that rule out $\phi^p$.
With probabilistic
transcription, this problem is quite common: if the
human transcribers fail to rule out $\phi^p$ (e.g., because the
correct and incorrect transcripts are perceptually
indistinguishable in the language of the transcribers), then the EM
algorithm will also never learn to rule out $\phi^p$.

EM's inability to learn zero-valued probabilities can be ameliorated
by using the segmental K-means algorithm~\cite{Juang1990}, which
bounds ${\mathcal L}(\theta')$ as
$\mathcal{L}(\theta')\ge
R(\theta,\theta')$:
\begin{align}
  R(\theta,\theta') &= \ln \pi(x,s^*(\theta),\phi^*(\theta)|\theta')\\
  s^*(\theta),\phi^*(\theta)&= \argmax_{s,\phi} \pi(s,\phi|x,\theta)
\end{align}
Given an initial parameter vector $\theta$, therefore, it is possible
to find a new parameter vector $\theta'$ with higher likelihood by
computing its maximum-likelihood senone sequence and phone sequence
$s^*(\theta),\phi^*(\theta)$, and by maximizing $\theta'$ with respect to
$s^*(\theta)$ and $\phi^*(\theta)$.
Maximizing
$R(\theta,\theta')$ rather than $Q(\theta,\theta')$ is useful for
probabilistic transcription because it reduces the importance of poor
phonetic transcripts.

\subsection{Using a Language Model During Training}
\label{sec:trainwithlm}

A PT contains significant amount of information beyond any single
transcript extracted from the PT. Motivated by this, the statistics
for the MAP estimation are accumulated from a lattice derived from the
cascade $\mathbf{AM} \circ \mathbf{HC} \circ \mathbf{PT}$, rather than
reducing the PT to its single best path. Though it is disadvantageous
to reduce a PT to its best path, it is nevertheless advantageous to
incorporate as much information as possible from the language model
during adaptation.  Define $\mathbf{G}$ to be an FST representing the
modeled phone bigram probability
$\pi(\phi^\ell|\theta)=\prod_{m=1}^M\pi(\phi_m^\ell|\phi_{m-1}^\ell,\theta)$.
By assumption, such information is not available from speech: we
assume that there is no transcribed speech in the target language.  A
reasonable proxy, however, can be constructed from text.

\begin{figure}
  \centerline{\includegraphics[width=5in]{../figs/fig_sloan.png}}
  \caption{A phonotactic language model (a bigram language model over
    phone sequences) can be trained using text data downloaded from
    Wikipedia (left), then converted into phone strings in the target
    language using a simple character-based grapheme-to-phoneme
    transducer (center).  In this example, the target language is
    Swahili.}
  \label{fig:wikitext}
\end{figure}

Fig.~\ref{fig:wikitext} shows text data downloaded from wikipedia in
Swahili, and a segment of a rule-based, character-by-character G2P for
the Swahili language~\cite{Hasegawajohnson15}.  By passing the former
through the latter, it is possible to generate synthetic phoneme
sequences in the target language.

\begin{figure}
  \centerline{\includegraphics[width=4in]{../figs/sloan2.png}}
  \caption{PER of the 1-best path: a measure of the quality of
    probabilistic transcriptions acquired from mismatched
    crowdsourcing.  Native transcriptions were available in six
    languages: Swahili (SW), Dutch (DT), Mandarin (MN), Urdu (UR),
    Arabic (AR), and Hungarian (HG).  Probabilistic transcriptions
    were decoded using three different methods per language: using a
    universal phoneme set (tallest bar in each language), using a
    phoneme set specific to the target language (middle bar in each
    language), and using a phonotactic language model derived from
    wikipedia texts (shortest bar in each language).}
  \label{fig:pt_decode_per}
\end{figure}

Phone error rate of the 1-best path through the mismatched
crwodsourcing confusion network are shown in
Fig.~\ref{fig:pt_decode_per}.  As shown, the use of a phonotactic
language model, derived from wikipedia text, reduced phone error rate
by about 10\% absolute, in each language.

Composing $\mathbf{PT}\circ \mathbf{G}$ is complicated, however, by
the presence of null transitions in the PT.  A null transition in
the PT matches a non-event in the language model, for which normal
FST notation has no representation. In order to compose the PT with
the language model, therefore, it is necessary to introduce a
special type of ``non-event'' symbol, here denoted ``\#2'', into the
language model (Fig.~\ref{fig:liu1}).  As shown in
Fig.~\ref{fig:liu1}, a language model ``non-event'' is a transition
that leaves any state, and returns to the same state (a self-loop).
Such self-loops, labeled with the special symbol ``\#2'' on both
input and output language, are added to every state in the
phonotactic language model (left-hand side of Fig.~\ref{fig:liu1}).
The probabilistic transcript, then, is augmented with the special
symbol ``\#2'' as the input-language symbol for every null-output
edge
 (output symbol is $\phi_m^\ell =\epsilon$).
\begin{figure}
  \centerline{\includegraphics[width=5in]{../figs/liu1.png}}
  \caption{Deletion edges in the probabilistic transcript (edges with
    the special null output symbol, $\epsilon$), required special
    handling in order to use information from a phonotactic language
    model.  As shown, a new type of null symbol, ``\#2'', was invented
    to represent the input for every PT edge with an $\epsilon$ output
    (right).  Such edges were only allowed to match with state
    self-loops, newly added to the language model (left) in order to
    consume such non-events in the transcript.}
  \label{fig:liu1}
\end{figure}


\subsection{Maximum {\em A Posteriori} Adaptation}
\label{sec:adaptation}

Training from PTs can be improved by starting from a multilingual ASR,
and adapting its parameters to PTs in the target language.  The
Bayesian framework for maximum {\em a posteriori} (MAP) estimation
has been widely applied to GMM and HMM parameter estimation problems
such as parameter smoothing and speaker
adaptation~\cite{gauvain1994maximum}.

Formally, for an unseen target language, denote its acoustic
observations $x = ( x_1^1, \ldots, x_{T}^L )$, and its acoustic model
parameter set as $\theta$, then the MAP parameters are defined as:
\begin{equation}
  \theta_{\mathrm{MAP}}  = \argmax_{\theta} \pi(\theta | x) 
= \argmax_{\theta} \pi( x | \theta ) \pi(\theta)
\label{eq:map}
\end{equation}
\noindent where $\pi(\theta)$ is the product of conjugate gradient
prior distributions, centered at the parameters of a cross-lingual
baseline GMM-HMM.  In a GMM-HMM, the acoustic model 
%\[
%\pi(x_t^\ell|s_t^\ell =j,\theta )=
%\sum_{k=1}^K c_{jk}
%\mathcal{N}
%\left(x_t^\ell|\mu_{jk},v_{jk}
%\right)
%\]
%\noindent
parameters $\theta=\left\{c_{jk},\mu_{jk},v_{jk}\right\}$ include
mixture weights, mean vectors, and variance vectors.  Maximum
likelihood trains these parameters by computing $\gamma_{\ell
  t}(j,k)=\pi(k|s_t^\ell =j,\phi^\ell,\theta)\pi(s_t^\ell
=j,\phi^\ell,\theta)$, then accumulating weighted average acoustic
frames with weights given by $\gamma_{\ell t}(j,k)$. Segmental K-means
quantizes $\pi(s_t^\ell
=j|\phi^\ell,\theta)\rightarrow\left\{0,1\right\}$ using forced
alignment, then proceeds identically.  MAP adaptation assigns, to each
parameter, a conjugate prior $\pi(\theta)$ with mode equal to
$\bar\theta$ (the parameters of the multilingual baseline), and with a
confidence hyperparameter $\tau_\theta$, resulting in re-estimation
formulae that are linearly interpolated between the baseline
parameters $\bar\theta$ and the statistics of the adaptation data, for
example:
\[
c_{jk}'=\frac{\tau_c\bar{c}_{jk}+\sum_{\ell,t}\gamma_{\ell t}(j,k)}
{\sum_{\kappa}\left(\tau_c\bar{c}_{j\kappa}+
  \sum_{\ell,t}\gamma_{\ell t}(j,\kappa)\right)},~~~
\mu_{jk}'=\frac{\tau_\mu\bar{\mu}_{jk}+\sum_{\ell,t}\gamma_{\ell t}(j,k)x_t^\ell}
   {\tau_\mu+\sum_{\ell,t}\gamma_{\ell t}(j,k)}
   %,~~~
%v_{jk}'=\frac{\tau_v\bar{v}_{jk}+\sum_{\ell,t}\gamma_{\ell t}(j,k)(x_t^\ell-\mu_{jk})^2}
%{\tau_v+\sum_{\ell,t}\gamma_{\ell t}(j,k)}
\]
%In our setting, the initial value of $\bar{\mu}_{jk}$ is obtained from
%the multilingual baseline model, and $\mu_{jk}'$ eventually converges
%to a model for the target language data.



\subsection{Neural Networks}

The NN acoustic model is
\[
\pi(x_t^\ell|s_t^\ell =j,\phi^\ell,\theta)\propto
y_t^\ell(j)=\frac{1}{c_j}\frac{\exp\left(w_j^Th_t(x,\theta_h)\right)}
{\sum_k \exp\left(w_k^Th_t(x,\theta_h)\right)}
\]
whose parameters $\theta=\left\{c_j,w_j,\theta_h\right\}$ include the
senone priors $c_j$, the softmax weight vectors $w_j$, and the
parameters defining the hidden nodes $h_t(x,\theta_j)$.  NNs are
trained by using a GMM-HMM to compute an initial senone posterior,
$\pi(s_t^\ell=j,x^\ell|\theta)$, then minimizing the cross-entropy
between the estimated senone posterior and the neural network output
$y_{t}^\ell(j)$.  The cross entropy is measured as
\begin{equation}
  H(S^\ell\Vert Y^\ell)=-\sum_{t=1}^T \sum_{j} \pi(s_t^\ell=j) \ln y_{t}^\ell(j)
  \label{eq:dnn_train}
\end{equation}
The gradient of Eq.~(\ref{eq:dnn_train}) with respect to its model
parameters is
\begin{equation}
  \nabla_\theta H(S^\ell\Vert Y^\ell)=-
  \sum_{t=1}^T\sum_j\frac{\pi(j,x^\ell|\theta)}{y_t^\ell(j)}
  \nabla_\theta y_t^\ell(j)
  \label{eq:dnn_dt}
\end{equation}
NN training with deterministic transcriptions is improved by
quantizing $\pi(s_t^\ell,x^\ell|\theta)\rightarrow\left\{0,1\right\}$
using forced alignment. Preliminary experiments showed that forced
alignment also improves the accuracy of NNs trained from probabilistic
transcriptions: the best path through the PT, and the best alignment
of the resulting senones to the waveform, were both computed using
forced alignment.  The resulting best senone string was used to train
a DNN using Eq.~(\ref{eq:dnn_dt}).



%%%%%%%%%%%%%%%%%%%%%%%%%%%%%%%%%%%%%%%%%%%%%%%%
\section{Experimental Methods}
\label{sec:methods}

Our goal is to train a phone recognition system for a given target language in which no native transcriptions are available. We assume that we have access to unspoken texts and to untranscribed audio in the target language, but not to transcribed audio.
% Eliminating the itemize here only to save space -- MH, 9/24/2015
%\begin{itemize}
%\item
Baseline multilingual systems are trained using native transcriptions from several different languages (not including the target language). Section~\ref{sec:mlbaseline} details multilingual GMM-HMM and DNN-based ASR systems with language-specific grammar models and Section~\ref{sec:selftraining} describes a semi-supervised baseline that uses unlabeled data from the target language.
%\item
Next, we adapt the parameters of the acoustic model of the above system using only probabilistic phone transcriptions in the target language derived from mismatched transcriptions. The construction of probabilistic phone transcriptions is described in Section~\ref{sec:MC} and the acoustic model adaptation is detailed in Section~\ref{sec:adaptation}.
%\end{itemize}
%% comments: the reference to the self-training section will need to 
%% change here, if (as I recommend) the *results* part of the
%% self-training section move into a new subsection within section 5 or 6.
%%
%% More generally, the cross-references in this section resolve to very
%% disparate parts of the manuscript (5.5, 4.1, 4.2, 3.3). I was 
%% expecting it to introduce the upcoming subsections within section 5.
%% I think this is a symptom of an overall lack of separation between
%% descriptions of the system, descriptions of the experiments run on
%% the system, and descriptions of the results of those experiments.
%% I've tried to leave some breadcrumbs in places where this was
%% especially evident to me, but I think MH is probably in the best
%% position to do the kind of heavy lifting required for such a re-org.   

\subsection{Data}
\label{sec:data}

Speech data were extracted from publicly available podcasts~\cite{SBS}
hosted in 68 different languages.  In order to generate test corpora
(in which it is possible to measure phone error rate), advertisements
were posted at the University of Illinois seeking native speakers
willing to transcribe speech in any of these 68 languages.  Of the ten
transcribers who responded, six people were each able to complete one
hour of speech transcription (the other four dropped out).  One
additional language was transcribed by workers recruited at $I^2R$ in
Singapore, yielding a total of seven languages with native
transcripts suitable for testing an ASR: Arabic (arb), Cantonese
(yue), Dutch (nld), Hungarian (hun), Mandarin (cmn), Swahili (swh) and
Urdu (urd).
{\color{blue} It is desirable to test the ideas in this paper with corpora larger
than one hour per language, but larger corpora involve problems
orthogonal to the purposes of this paper, e.g., the Babel corpora
are telephone speech, and therefore contain far more acoustic
background noise than the podcast corpora used in this paper.}

The podcasts contain utterances interspersed with segments of music
and English. A GMM-based language identification system was
developed in order to isolate
regions that correspond mostly to the target language, which
were then split into 5-second
segments to enable easy labeling by the native transcribers.
%and more importantly to allow for the collection of mismatched
%transcripts that required the speech segments to be short. To
%further check that only speech clips in the target language were
%retained, the
Native transcribers were asked to omit any 5-second
clips that contained significant music, noise, English,
or speech from multiple speakers. Resulting transcripts
covered 45 minutes of speech in Urdu and 1
hour of speech in the remaining six languages. The orthographic
transcripts for these clips were then converted into phonemic
transcripts using language-specific dictionaries and G2P mappings
(these resources are detailed in Section~\ref{sec:mlbaseline}). For
each language, we chose a random 40/10/10 minutes split into training,
development and evaluation sets.
%Table~\ref{tab:data} describes the
%resulting training, development and evaluation sets.
%\begin{table}[t]
%\centering
%\begin{tabular}{|c||ccc|}\hline
%Language  & \multicolumn{3}{c|}{Speech (\# phones)}\\
%(ISO 639-3) & Train & Dev & Eval \\ \hline\hline
%arb & 32486 & 8208 & 8191 \\
%yue & 32693 & 6860 & 8638 \\
%nld & 27314 & 6943 & 6582 \\
%hun & 29461 & 7873 & 7474 \\
%cmn & 29461 & 8244 & 7035 \\
%swh & 28571 & 7658 & 7441 \\
%urd & 21275 & 5808 & 3689 \\\hline
%\end{tabular}
%\vspace*{1mm}
%\caption{Data statistics, seven languages, \# 
%phones in the train/dev/eval sets.}
%\label{tab:data}
%\end{table}

\subsection{Mismatched Crowdsourcing}
\label{sec:methodsmc}

Mismatched transcriptions were collected from crowd workers (Turkers)
on Amazon Mechanical Turk~\cite{MTurk} for all the data listed in
Table~\ref{tab:data}.  The crowdsourcing task setup is described
in~\cite{JHJ15b}.  Each 5-sec speech segment was further split into 4
non-overlapping segments to make the non-native listening task
easier. The crowdsourcing task was set up as described
in~\cite{JHJ15b}; briefly, the short segments were played to Turkers,
who transcribed what they heard (typically in the form of nonsense
syllables) using English orthography. Each short recording segment was
transcribed by 10 distinct Turkers. More than 2500 Turkers
participated in these tasks, with roughly 30\% of them claiming to
know only English. (Spanish, French, German, Japanese, Chinese were
some of the other languages listed by the Turkers.)


\subsection{EEG Recording and Analysis}
%% TODO (Majid): add standard EEG recording methods boilerplate.
To compute distinctive feature weights used in estimating the misperception
transducer as shown in Eqs.~\ref{eq:dfdist} and~\ref{eq:eegdist},
recordings of cortical activity in response to non-native phones were
made using EEG. Signals were acquired using a BrainVision XXX EEG system 
with XXX channels and XXX sampling rate and XXX online filtering.
All methods were approved by the University of Washington Institutional
Review Board.

Auditory stimuli used to evoke the electrocortical responses comprised
consonant-vowel (CV) syllables representing three languages: English,
Dutch and Hindi. The inclusion of only two non-English languages in the
auditory stimuli was dictated by the relatively high number of
repetitions required to achieve good signal-to-noise ratio from averaged
EEG recordings. The choice of Dutch and Hindi was governed by their
inclusion in the SBS subset used to train the misperception G2P as
described in Section~\ref{sec:MC}, and the relative similarities
between their phoneme inventories and the phoneme inventory of English.
In particular, we chose languages with relatively many (Hindi) or few
(Dutch) ``many-to-one mappings'' between the non-English phoneme
inventory and the phoneme inventory of English, estimated based on 
distinctive feature representations of the phonemes in each language 
(as given in the PHOIBLE database~\cite{PHOIBLE}). Such 
``many-to-one mappings'' are expected to pose a problem for the 
non-native transcription task being modeled by the misperception 
transducer, so to test the limits of our design we chose languages that 
differed greatly in this property. 
%% rest of this paragraph could probably be cut if necessary
Note that, although Hindi podcasts were not included in the SBS training
data described in Section~\ref{sec:data}, colloquial spoken Hindi and
Urdu are extremely similar phonologically~\cite{Kachru90}, and
considering that the auditory stimuli for the EEG portion of this
experiment are simple CV syllables, it is reasonable to consider Hindi
and Urdu as equivalent for the purpose of computing feature weights for
the misperception transducer.

%% This next section, including tab:m2o, seems like more detail than
%% is necessary; the explanation above as to **why** we chose Hindi and
%% Dutch (combined with tab:eegphones) seems sufficient here, without
%% going into the extra details of **how** we settled on those two.
Language similarity was defined as the number of many-to-one mappings
($N_{M2O}(\mathbb{\Phi})$) between the English phoneme inventory
($\mathbb{\Psi}$) and the non-native phoneme inventory $\mathbb{\Phi}$.
Using distinctive feature representations of the phonemes in each
inventory (as given in the PHOIBLE database \cite{PHOIBLE}), a
many-to-one mapping was defined by finding, for each
non-native phoneme $\phi$, the English phoneme $\psi^*(\phi)$ to which
it is most similar:
\begin{equation}
  \psi^*(\phi) = \argmin \sum_k \delta\left(f_k(\psi)\ne f_k(\phi)\right)
\end{equation}
The number of many-to-one collisions is then defined as
\begin{equation}
  N_{M2O}(\mathbb{\Phi})=\frac{1}{|\mathbb{\Psi}|}\sum_{\phi_1\ne\phi_2}
  \delta\left(\psi^*(\phi_1)=\psi^*(\phi_2)\right)
\label{eq:m2o}
\end{equation}
where $|\mathbb{\Psi}|$ is the size of the English phoneme inventory.
The frequency of many-to-one mappings is listed in
Table~\ref{tab:m2o} for several languages.

\begin{table}
  \centerline{\begin{tabular}{|lr|lr|lr|}\hline
    $\mathbb{\Phi}$ & $N_{M2O}(\mathbb{\Phi})$ &
    $\mathbb{\Phi}$ & $N_{M2O}(\mathbb{\Phi})$ &
    $\mathbb{\Phi}$ & $N_{M2O}(\mathbb{\Phi})$ \\ \hline
    spa & 0.862 & yue & 1.280 & cmn & 1.531 \\
    por & 1.152 & jpn & 1.333 & amh & 1.844 \\
    nld & 1.182 & vie & 1.393 & hun & 1.857 \\
    deu & 1.258 & kor & 1.429 & hin & 2.848 \\\hline
  \end{tabular}}
  \caption{Frequency of many-to-one mappings $N_{M2O}(\mathbb{\Phi})$
    between phoneme inventory $\mathbb{\Phi}$ and the inventory of
    English. Languages are represented by their ISO 639-3 codes.}
  \label{tab:m2o}
\end{table}

To construct the auditory stimuli, two vowels and several consonants
were selected from the phoneme inventory of each language (18 consonants
for English, 17 for Dutch, and 19 for Hindi). Choice of consonants was
made so as to emphasize differences in the many-to-one relationships
between English-Dutch and English-Hindi, while maintaining roughly equal 
numbers of consonants for each language. The consonants chosen for each 
language are given in Table~\ref{tab:eegphones}; the vowels chosen were 
the same for all three languages (/a/ and /e/).

\begin{table}
  \centering
  \setlength{\tabcolsep}{0.3em}
  \setlength\extrarowheight{2pt}
  \begin{tabular}{|l||cc|cccc|cc|c|cccc|cc|c|c|c|c|c|c|c|c|c|cc|c|c|c|}\hline
    Language & \multicolumn{29}{c|}{Consonant phones used in the EEG experiment}\\ \hline
    eng & \multicolumn{2}{c|}{p} & \multicolumn{4}{c|}{t} & \multicolumn{2}{c|}{k} & \textipa{p\super h} & \multicolumn{4}{c|}{\textipa{t\super h}} & \multicolumn{2}{c|}{\textipa{k\super h}} & \textipa{tS} & \textipa{tS\super h} & f & \textipa{T} & \textipa{S} & v & \textipa{D} & z & m & \multicolumn{2}{c|}{n} & l & \textipa{\*r} & \\ \hline
    nld &  \multicolumn{2}{c|}{p} & \multicolumn{4}{c|}{t} & \multicolumn{2}{c|}{\textipa{G}} & \textipa{p\super h} & \multicolumn{4}{c|}{\textipa{t\super h}} & \multicolumn{2}{c|}{\textipa{k\super h}} & & \textipa{tS\super h} & f & & \textipa{S} & v & & z & m & \multicolumn{2}{c|}{n} & l & \textipa{\;R} & j \\ \hline
    hin &  p & b & \textipa{\|[t} & \textipa{\|[d} & \textipa{\:t} & \textipa{\:d} & k & \textipa{g} & \textipa{b\super H} & \textipa{\|[t\super h} & \textipa{\:t\super h} & \textipa{\|[d\super H} & \textipa{\:d\super H} & \textipa{k\super h} & \textipa{g\super H} & & & & & & \textipa{V} & & & m & \textipa{\|[n} & \textipa{\:n} & & & \\ \hline
  \end{tabular}
  \caption{Consonant phones used in the EEG experiment represented using
  IPA. Vertical alignment of cells suggests many-to-one mappings
  expected based on distinctive feature values from PHOIBLE.}
  \label{tab:eegphones}
\end{table}

Two native speakers of each language (one male and one female) were
recorded (XXX sample rate, bit depth) 
speaking multiple repetitions of the set of CV syllables for
their language. Three tokens of each unique syllable were excised from
the raw recordings (XXX downsampled, XX RMS normalized)
Recorded syllables had an average duration of 400~ms, and were presented
via headphones to one monolingual American English listener.
The stimuli were presented in XXX blocks of XXX minutes per block, for a
total of XXX minutes.  Syllables were presented in random order with an
inter-stimulus interval of 350~ms. XXX repetitions of each syllable
were presented, for a grand total of XXX syllable presentations.

EEG recordings were divided into XXX ms epochs beginning XXX ms before
each syllable onset.
The epoched data were coded with a subset of distinctive features
that minimally defined the phoneme contrasts of the English consonants.
Where more than one choice of features was sufficient to define those
contrasts, preference was given to features that reflect differences
in temporal as opposed to spectral features of the consonants, due to
the high fidelity of EEG at reflecting temporal envelope properties of 
speech.~\cite{Liberto15} The final set of features chosen was:
continuant, sonorant, delayed release, voicing, aspiration, labial,
coronal, and dorsal.
% In other words, differences in the temporal amplitude envelope of
% consonants have a better chance of being recoverable after being
% filtered through a human auditory system and cortex than do differences
% that are purely spectral in nature; to the extent that spectral
% information in speech is preserved in an EEG signal, it will have been
% transformed to be spatially coded across populations of neurons.

Epoched and feature-coded EEG data {\em for the English syllables only}
were used to train a support vector machine classifier for each feature.
The classifiers were then used (without re-training) to classify the
EEG responses to the Dutch and Hindi syllables.
Fig.~\ref{fig:eeg_svm_eers} shows equal error rates of these
classifiers when applied to the three languages.

\begin{figure}
  \centerline{\includegraphics[width=5in]{../figs/diliberto_svmresults.png}}
  \caption{Classifiers were trained to observe EEG signals, and to
    classify the distinctive features of the phoneme being audited.
    Equal error rates are shown for English (the language used in
    training; train and test data did not overlap), Dutch, and Hindi
    (not used in training).}
  \label{fig:eeg_svm_eers}
\end{figure}

Eq.~\ref{eq:dfdist} defines a log-linear model of $\rho(\psi|\phi)$, the
probability that a non-English phoneme $\phi$ will be perceived as English
phoneme $\psi$.  Denote by $\rho_U(\psi|\phi)$ the model of
Eq.~\ref{eq:dfdist} with uniform weights for all distinctive features
($w_k=\alpha$, a tunable constant).  Denote by $\rho_{EEG}(\psi|\phi)$ the
same model, but with weights $w_k$ derived from EEG measurements
(Eq.~\ref{eq:eegdist}).  Fig.~\ref{fig:eeg_confusions} shows these
two confusion matrices: $\rho_U(\psi|\phi)$ on the left,
$\rho_{EEG}(\psi|\phi)$ on the right. The entropy of the
uniform weighting, $\rho_U(\psi|\phi)$, is too low: when a Dutch
phoneme $\phi$ has a nearest-neighbor $\psi^*(\phi)$ in English, then
few other phonemes are considered to be possible confusions.
$\rho_{EEG}(\psi|\phi)$ has a very different problem: since distinctive
feature classfiers have been trained for only a small set of
distinctive features,
there are large groups of phonemes whose confusion
probabilities can not be distinguished (giving the figure its
block-diagonal structure).  The faults of both models can be
ameliorated by interpolating them in some way, e.g., by computing the
linear interpolation
$\rho_I(\psi|\phi)=a\rho_U(\psi|\phi)+(1-a)\rho_{EEG}(\psi|\phi)$ for some
constant $0\le a\le 1$.

\begin{figure}
  \centerline{
    \includegraphics[width=3in]{../figs/mirbagheri_dist_features.png}
    \includegraphics[width=3in]{../figs/mirbagheri_dist_eeg.png}
  }
  \caption{Phoneme confusion probabilities between English phonemes
    (column) and Dutch phonemes (row) using models in which the log
    probability is proportional to distance between the corresponding
    distinctive features.  Left: all features have the same
    weight.  Right: feature weights equal negative log error rate of
    EEG signal classifiers.}
  \label{fig:eeg_confusions}
\end{figure}

\subsection{Multilingual Baselines}
\label{sec:mlbaseline}

The goal of building a multilingual system is two-fold.
One is to setup a baseline for generalizing to an unseen
language without any labeled audio corpus.  The other
is have the baseline serve as a starting point for
adaptation.

The dataset consists of 40 minutes of labeled audio for training,
10 minutes for development, 10 minutes for testing
for each language.
The orthographic transcriptions are converted into
phonetic transcriptions in the following steps.
Beginning with a list of the IPA symbols used in canonical descriptions
of all seven languages,
any symbol appearing in only one languages was merged with a symbol
differing by only one distinctive feature; this process proceeded until 
each remaining phone symbol is represented in at least two languages.
English words are identified and converted to phones with
an English G2P trained using CMUdict~\cite{Lenco15}.
We take the canonical pronunciation of a word if the word
appears in a lexicon,
otherwise estimate the word's pronunciation using a G2P.
The Arabic dictionary is from the Qatari Arabic Corpus~\cite{Elmahdy14},
the Dutch dictionary is from CELEX v2~\cite{Baayen96},
the Hungarian dictionary was provided by BUT~\cite{Grezl14},
the Cantonese dictionary is from $I^2R$,
the Mandarin dictionary is from CALLHOME~\cite{Canavan96},
and the Urdu and Swahili G2Ps were compiled from
rule-based descriptions of the orthographic systems in those
two languages~\cite{Hasegawajohnson15}.

Each HMM was trained with data from six languages, tuned
(hyperparameters) on the development set of the seventh language, and
tested on the evaluation set of the seventh language.  The lexicon of
the target language was not used during testing, but two types of
language-dependent specialization were allowed.  In the first type of
specialization, the universal phone set was restricted at test time to
output only phones in the target language.  In the second type of
specialization, a target-language phone bigram language model was
trained using phone sequences converted from text.  The texts were
collected from Wikipedia articles linked from the main page of each
language crawled once per day over four months.
Table~\ref{tab:results} compares results using universal phone set and
phone language model to those obtained using language-dependent phone
set and phone language model.  Without a language specific phone set
and phone language model, it is hard for a multilingual system to
generalize to an unseen language.  This is true even if the system has
seen closely related languages such as Mandarin when tested on
Cantonese.  As an oracle experiment, we also train language dependent
HMMs for each individual language with 40 minutes of labeled audio.
Results are shown in Table~\ref{tab:results}.

\begin{table*}
\begin{center}
\begin{tabular}{|c|c|c|cccc|}
\hline
data & acoustic & language & yue & hun & cmn & swh \\
 & model & model &  & & & \\
\hline
multilingual & GMM & multilingual & 79.64 (79.83) & 77.13 (77.85) & 83.28 (82.12) & 82.99 (81.86) \\
multilingual & NN & multilingual & 78.62 (77.58) & 75.98 (76.44) & 81.86 (80.47) & 82.30 (81.18) \\
multilingual & GMM & text & 68.40 (68.35) & 68.62 (66.90) & 71.30 (68.66) & 63.04 (64.73) \\
multilingual & NN & text & 66.54 (65.28) & 66.08 (66.58) & 65.77 (64.80) & 64.75 (65.04) \\
\hline
monolingual & GMM & transcript & 32.77 (34.61) & 39.58 (39.77) & 32.21 (26.92) & 35.33 (46.51) \\
monolingual & NN & transcript & 27.67 (28.88) & 35.87 (36.58) & 27.80 (23.96) & 34.98 (41.47) \\
\hline
\end{tabular}
\caption{\label{tab:results} PERs of unadapted multilingual systems on
  the evaluation sets along with monolingual systems.  PERs on the
  development sets are in parentheses.  Text-based language models are
  trained using phone sequences computed by applying a G2P to
  independent wikipedia texts in the target language. Transcript-based
  language models are trained using phone sequences computed by
  applying a G2P to native transcripts of the training data.}
\end{center}
\end{table*}

From the comparison of different baseline systems, we can reach the
following conclusions.  First, the standard speech pipeline is able to
train speech recognizers using SBS data.  Even with only 40 minutes of
training data, a NN is able to outperform a GMM.  Second,
however, the standard speech pipeline generalizes well to languages
that were seen in the training corpus, but performs poorly on unseen
languages.  Using a language-specific phonotactic language model gives
significant improvement over the language-independent phonotactic
model, but nevertheless significantly underperforms a system that
has seen the test language during training.  


\subsection{MAP Adaptation to Probabilistic Transcripts}
\label{sec:ptadapt}

The baseline and the adapted models were implemented using
Kaldi~\cite{Kaldi2011}. In order to efficiently carry out the required
operations on the cascade $\mathbf{AM}\circ\mathbf{HC}\mathbf{PT}$, we
carefully designed $\mathbf{PT}$ as an acceptor defined as
$\mathrm{proj}_{\mathrm{input}} (\widehat{PT})$, where $\widehat{PT}$
is a WFST mapping phone sequences to English letter sequences obtained
as a cascade of WFSTs modeling the distributions shown in
Equation~\ref{eq:PT}, and $\mathrm{proj}_{\mathrm{input}}$ refers to
projecting onto the input labels. For the purposes of computational
efficiency, the cascade for $\widehat{PT}$ includes an additional WFST
restricting the number of consecutive deletions of phones and
insertions of letters (to a maximum of 3 in our experiments). We use
two additional disambiguation symbols~\cite{mohri2008speech}, apart
from the ones used in typical Kaldi recipes, to determinize these
insertions and deletions in $\widehat{PT}$. MAP adaptation for the
acoustic model was carried out for a number of iterations (12 for yue
\& cmn, 14 for hun \& swh, with a re-alignment stage in iteration 10).


%%%%%%%%%%%%%%%%%%%%%%%%%%%%%%%%%%%%%%%%%%%%%%%%%%%%%%%%%%
\section{Experimental Results}
\label{sec:results}

This section reports two types of results.  First,
subsections~\ref{s6:mc} and~\ref{ssec:eeg} report improvements in the
quality of probabilistic transcription using information acquired from
text-based phone language models and EEG signals, respectively.
Second, subsections~\ref{s6:mlbaseline} and~\ref{ssec:asr} reports the
accuracy of cross-lingual ASR and PT-adapted ASR, respectively.


%\subsection{Mismatched Crowdsourcing}
\label{s6:mc}

The quality of a probabilistic transcript derived from mismatched
crowdsourcing is significantly improved by using a phone language
model during the decoding process ($\rho(\phi)$ in Eq.~(\ref{eq:PT})).
Phone
language models for each target language were computed from Wikipedia
texts using the methods described in Sec.~\ref{sec:trainwithlm}.
Label phone error rate
(LPER) of the 1-best path through the resulting PTs are shown in
Table~\ref{fig:pt_decode_per}, computed with reference to a native
transcript in each language.  As shown, the use of a phone
language model, derived from Wikipedia text, reduces LPER by about 10\%
absolute, in each language.

\begin{table}
\centerline{\begin{tabular}{|c||c|c|c|c|c|c|}\hline
    Method & nld & cmn & urd & arb & hun &swh \\\hline
    Universal set & 87.4 & 88.86 & 97.95 & 79.04 & 92.87 & 88.56 \\
    Target set & 78.12 & 87.4 & 87.81 & 66.39 & 84.78 & 59.83 \\
    Phone bigram & 70.43 & 70.88 & 64.67 & 65.29 & 63.98 & 50.45 \\\hline
\end{tabular}}
\vspace*{1mm}
\caption{Label phone error rate (LPER) of probabilistic transcripts
  for universal phone set, target-language phone set, text-based
  phone bigram.}
\label{fig:pt_decode_per}
\end{table}

LPER of the 1-best path does not
accurately reflect the extent of information in the PTs that can be
leveraged during ASR adaptation.  Consider, for example, the four
Urdu phones~\ipa{[p,p\textsuperscript{h},b,b\textsuperscript{H}]}.  An attentive
English-speaking transcriber must choose between the two letters
$<$p,b$>$ in order to represent any of these four phones.  The
misperception G2P therefore maps the letters $<$p,b$>$ into a
distribution over the phones~\ipa{[p,p\textsuperscript{h},b,b\textsuperscript{H}]}.
There is no reason to expect that the maximizer of
$\rho(\phi|\lambda)$ is correct, but there is good reason to expect
the correct answer to be a member of a short $N$-best list ($N\le 4$
phones/grapheme).  A fuller picture is therefore obtained by
pruning the PT to
a small number of paths, then searching for the most correct path
in the pruned PT.  One useful metric is entropy per segment, defined
as $H^{\ell}(\Phi)=-\frac{1}{M}\sum_{m=1}^M\sum_{u} \log_2\rho_{\Phi_m^\ell}(u)$,
e.g., a PT in which every segment has two equally probable options 
would measure $H^\ell(\Phi)=1$.
Fig.~\ref{fig:listPER}
shows the trend of LPER (for three languages) obtained by pruning
the PT at several different levels of $H^{\ell}(\Phi)$.
LPER rates drop significantly across all languages within 1
bit of entropy per phone, illustrating the extent of information
captured by the PTs.

\begin{figure}[t!]
  \begin{center}
\begin{tikzpicture}[
    scale=\mytikzscale,
    every node/.style={transform shape}]
\draw[<->] (0,-0.5) to (0,4.5);
\draw[<->] (-0.5,0) to (10.5,0);
\draw (-0.1,0.5) to (0.1,0.5);
\node at (-0.25,0.5) {35};
\draw (-0.1,1.0) to (0.1,1.0);
\node at (-0.25,1.0) {40};
\draw (-0.1,1.5) to (0.1,1.5);
\node at (-0.25,1.5) {45};
\draw (-0.1,2.0) to (0.1,2.0);
\node at (-0.25,2.0) {50};
\draw (-0.1,2.5) to (0.1,2.5);
\node at (-0.25,2.5) {55};
\draw (-0.1,3.0) to (0.1,3.0);
\node at (-0.25,3.0) {60};
\draw (-0.1,3.5) to (0.1,3.5);
\node at (-0.25,3.5) {65};
\draw (-0.1,4.0) to (0.1,4.0);
\node at (-0.25,4.0) {70};
\draw (1,-0.1) to (1,0.1);
\node at (1,-0.25) {0.1};
\draw (2,-0.1) to (2,0.1);
\node at (2,-0.25) {0.2};
\draw (3,-0.1) to (3,0.1);
\node at (3,-0.25) {0.3};
\draw (4,-0.1) to (4,0.1);
\node at (4,-0.25) {0.4};
\draw (5,-0.1) to (5,0.1);
\node at (5,-0.25) {0.5};
\draw (6,-0.1) to (6,0.1);
\node at (6,-0.25) {0.6};
\draw (7,-0.1) to (7,0.1);
\node at (7,-0.25) {0.7};
\draw (8,-0.1) to (8,0.1);
\node at (8,-0.25) {0.8};
\draw (9,-0.1) to (9,0.1);
\node at (9,-0.25) {0.9};
\node at (-1,2) {LPER};
\node at (5,-0.75) {Entropy of the PT (bits per segment)};
\node[rectangle,draw=black] at (12.5,2) {\begin{tabular}{l}-s- = Swahili\\-c- = Mandarin\\-h- = Hungarian\end{tabular}};
\node at (0.4,1.6200000000000003) {s};
\node at (0.37,3.9) {c};
\node at (0.3,3.2700000000000005) {h};
\node at (0.7000000000000001,1.4799999999999998) {s};
\node at (0.8400000000000001,3.7) {c};
\node at (0.68,3.16) {h};
\node at (1.2,1.36) {s};
\node at (1.37,3.5400000000000005) {c};
\node at (1.24,3.03) {h};
\node at (1.9900000000000002,1.1600000000000001) {s};
\node at (2.0,3.35) {c};
\node at (2.3200000000000003,2.78) {h};
\node at (2.9699999999999998,1.0100000000000002) {s};
\node at (2.7600000000000002,3.16) {c};
\node at (3.77,2.5700000000000003) {h};
\node at (4.37,0.8100000000000002) {s};
\node at (4.43,2.84) {c};
\node at (5.380000000000001,2.3899999999999997) {h};
\node at (5.21,0.7299999999999998) {s};
\node at (6.5,2.5100000000000002) {c};
\node at (6.34,2.2700000000000005) {h};
\node at (6.24,0.6299999999999997) {s};
\node at (7.57,2.37) {c};
\node at (7.45,2.19) {h};
\node at (7.4399999999999995,0.5600000000000002) {s};
\node at (8.8,2.2299999999999995) {c};
\node at (8.7,2.09) {h};
\node at (10.0,0.38999999999999985) {s};
\node at (10.2,2.0799999999999996) {c};
\node at (10.0,1.9899999999999998) {h};
\draw (0.4,1.6200000000000003) to (0.7000000000000001,1.4799999999999998);
\draw (0.7000000000000001,1.4799999999999998) to (1.2,1.36);
\draw (1.2,1.36) to (1.9900000000000002,1.1600000000000001);
\draw (1.9900000000000002,1.1600000000000001) to (2.9699999999999998,1.0100000000000002);
\draw (2.9699999999999998,1.0100000000000002) to (4.37,0.8100000000000002);
\draw (4.37,0.8100000000000002) to (5.21,0.7299999999999998);
\draw (5.21,0.7299999999999998) to (6.24,0.6299999999999997);
\draw (6.24,0.6299999999999997) to (7.4399999999999995,0.5600000000000002);
\draw (7.4399999999999995,0.5600000000000002) to (10.0,0.38999999999999985);
\draw (0.37,3.9) to (0.8400000000000001,3.7);
\draw (0.8400000000000001,3.7) to (1.37,3.5400000000000005);
\draw (1.37,3.5400000000000005) to (2.0,3.35);
\draw (2.0,3.35) to (2.7600000000000002,3.16);
\draw (2.7600000000000002,3.16) to (4.43,2.84);
\draw (4.43,2.84) to (6.5,2.5100000000000002);
\draw (6.5,2.5100000000000002) to (7.57,2.37);
\draw (7.57,2.37) to (8.8,2.2299999999999995);
\draw (8.8,2.2299999999999995) to (10.2,2.0799999999999996);
\draw (0.3,3.2700000000000005) to (0.68,3.16);
\draw (0.68,3.16) to (1.24,3.03);
\draw (1.24,3.03) to (2.3200000000000003,2.78);
\draw (2.3200000000000003,2.78) to (3.77,2.5700000000000003);
\draw (3.77,2.5700000000000003) to (5.380000000000001,2.3899999999999997);
\draw (5.380000000000001,2.3899999999999997) to (6.34,2.2700000000000005);
\draw (6.34,2.2700000000000005) to (7.45,2.19);
\draw (7.45,2.19) to (8.7,2.09);
\draw (8.7,2.09) to (10.0,1.9899999999999998);
  \end{tikzpicture}
\end{center}

  \vspace*{-0.5cm}
  \caption{LPER plotted against entropy rate estimates of phone sequences in three different languages.}
\label{fig:listPER}
\end{figure}


\label{ssec:eeg}

\newcommand{\specialcell}[2][c]{%
  \begin{tabular}[#1]{@{}c@{}}#2\end{tabular}}

Epoched and feature-coded EEG data {\em for the English syllables
only} were used to train a support vector machine classifier for each
distinctive feature.  The classifiers were then used (without
re-training) to classify the EEG responses to the Dutch and Hindi
syllables.  Fig.~\ref{fig:eeg_svm_eers} shows equal error rates of
these classifiers when applied to the three languages. 
EER of the classifier when applied to English phones is comparable to
those reported in~\cite{Liberto15}, the only prior work to attempt a
recognition of speech phonemes from EEG of the listener.

\begin{figure}
  \centerline{\includegraphics[width=\columnwidth]{../figs/eer-barplot/eer-barplot.pdf}}
  \vspace*{-0.3cm}
  \caption{Classifiers were trained to observe EEG signals, and to
    classify the distinctive features of the phone being heard.  Equal
    error rates are shown for English (the language used in training;
    train and test data did not overlap), Dutch, and Hindi.  Dashed
    line shows chance=50\%.}
  \label{fig:eeg_svm_eers}
\end{figure}

Eq.~(\ref{eq:dfdist}) defines a log-linear model of $\rho(\psi|\phi)$,
the probability that a non-English phoneme $\phi$ will be perceived as
English phoneme $\psi$.  Denote by $\rho_U(\psi|\phi)$ the model of
Eq.~(\ref{eq:dfdist}) with uniform binary weights for all distinctive
features. Denote by $\rho_{EEG}(\psi|\phi)$ the same model, but with
weights $w_k$ derived from EEG measurements (Eq.~(\ref{eq:eegdist})).
Fig.~\ref{fig:eeg_confusions} shows these two confusion matrices:
$\rho_U(\psi|\phi)$ on the left, $\rho_{EEG}(\psi|\phi)$ on the
right. The entropy of the binary weighting, $\rho_U(\psi|\phi)$, is
too low: when a Dutch phoneme $\phi$ has a nearest-neighbor
$\psi^*(\phi)$ in English, then few other phonemes are considered to
be possible confusions.  $\rho_{EEG}(\psi|\phi)$ has a very different
problem: since distinctive feature classifiers have been trained for
only a small set of distinctive features, there are large groups of
phonemes whose confusion probabilities can not be distinguished
(giving the figure its block-matrix structure).  The faults of both
models can be ameliorated by averaging them in some way, e.g., by
computing the linear interpolation
$\rho_I(\psi|\phi)=(1-\alpha)\rho_U(\psi|\phi)+\alpha\rho_{EEG}(\psi|\phi)$
for some constant $0\le\alpha\le 1$.

\begin{figure}
  \centerline{
    \includegraphics[width=\columnwidth]{../figs/confusion-matrix/confusion-matrices.pdf}
  }
    \vspace*{-0.3cm}
  \caption{Phone confusion probabilities between English and Dutch
    phones using models in which the negative log
    probability is proportional to unweighted or weighted
    distance between the corresponding
    distinctive feature vectors.  Left: unweighted.
    Right: feature weights equal negative log confusion
    probability of EEG signal classifiers.}
  \label{fig:eeg_confusions}
\end{figure}

In order to evaluate the effectiveness of the EEG-induced
misperception transducer we looked at the LPER of mismatched
crowdsourcing for Dutch when performed using 1)~a multilingual
misperception model $\rho(\lambda|\phi)$ (the machine translation
model described in Sec.~\ref{sec:MC}), 2)~feature-based misperception
transducer computed using binary weighting, $\rho_U(\psi|\phi)$, or 3)
EEG-induced transducer combined with the feature-based transducer,
$\rho_I(\psi|\phi)$.  Both method (2) and method (3)
required the use of a G2P in order to compute $\rho(\lambda|\psi)$:
the Dutch G2P was estimated using the CELEX database, while the Hindi
G2P was estimated using the zero-resource knowledge-based method
described in Sec.~\ref{sec:trainwithlm}.  The constant $\alpha=0.29$ was
chosen as the average of the values selected by all folds in a
leave-one-out cross-validation.  LPER of the multilingual model was
70.43\% (as shown in Table~\ref{fig:pt_decode_per}), of
the feature-based model, 69.44\%, and of the EEG-interpolated model,
68.61\%.




%\subsection{Cross-Lingual Baseline}
\subsection{ASR Results}
\label{s6:mlbaseline}

Tables~\ref{tab:ptresult} and~\ref{tab:dnnresult} present phone error
rates (PERs) for four different languages.  The first
column shows the phone error rate (PER) of monolingual topline
systems: evaluation test results are followed by development test
results in parentheses.  The column titled {\sc CL} lists
cross-lingual baseline error rates.  The column labeled {\sc ST} lists
the PERs of self-trained ASR systems.  The column headed {\sc
  PT-adapt} in Table~\ref{tab:ptresult} lists PERs from {\sc CL} ASR
systems that have been adapted to PTs derived from
mismatched crowdsourcing. Phone error rates are reported instead of
word error rates because, in order to compute a word error rate, it
is necessary to have either native transcriptions in the target
language (thereby permitting the training of a grapheme-based
recognizer) or a pronunciation lexicon in the target language.
These resources are used by the monolingual topline, but not by any
of the baseline or experimental systems.

The monolingual ASR is trained using only 40 minutes of audio and
transcript data per language, but performs reasonably well (31.58\%
average PER, NN-HMM).  The cross-lingual ASRs, however, perform poorly.
Using a text-based phone bigram (denoted {\sc TEXT}) gives significant
improvement over a cross-language phone bigram (denoted {\sc CL}),
but significantly underperforms a system that has seen the test
language during training.  This is true even if the system has seen
closely related languages during training: the Cantonese cross-lingual
system has seen Mandarin during training, and the Mandarin system has
seen Cantonese during training, but neither system is able to
generalize well from its six training languages to its test language.
Three different types of discriminative training were
tested.  MMI performs consistently worse than MPE and sMBR, and is
therefore not listed in Table~\ref{tab:ptresult}.  Averaged across
all languages and systems shown in Table~\ref{tab:ptresult}, the
development-test PERs of ML, MPE, and sMBR training are 73.43\%,
73.04\%, and 72.98\% respectively; differences are not statistically
significant, therefore only the ML system was tested on evaluation
test data.


\subsection{ASR Trained Using Probabilistic Transcriptions}
\label{ssec:asr}

\setlength{\tabcolsep}{0.37cm}
\begin{table*}[t]
\centerline{\begin{tabular}{| c || c | c | c | c | c |}\hline
Language &  Multilingual & Self-training & \multicolumn{3}{ |c| }{Mult-L + PT adaptation}  \\\cline{4-6}
Code & ({\sc Mult-L}) & ({\sc ST}) &  ({\sc PT-adapt}) & \% Rel. redn & \% Rel. redn\\\cline{4-6}
 &&&& \% over {\sc Mult-L} & \% over {\sc ST}\\
\hline
\multicolumn{6}{|l|}{GMM-HMM} \\\hline
yue & 68.40 (68.35) & &  \textbf{57.20 (56.57)} &  16.4 (17.1) & 10.3 (9.2) \\
hun & 68.62 (66.90) & &   \textbf{56.98 (57.26)} & 16.9 (14.3) & 10.2 (9.9) \\
cmn & 71.30 (68.66) & &   \textbf{58.21 (57.85)} &  18.4 (15.7) & 10.3 (9.7) \\
swh & 63.04 (64.73) & &   \textbf{44.31 (48.88)} & 29.6 (24.6) & 24.7 (18.4) \\
\hline\hline
\multicolumn{6}{|l|}{NN-HMM} \\\hline
yue & 66.54 (65.28) & 63.79 (62.46) &&& \\
hun & 66.08 (66.58) & 63.53 (63.50) &   \textbf{xxx (58.4)} & xxx (12.3) & xxx (8.0) \\
cmn & 65.77 (64.80) & 64.90 (64.00) &   \textbf{xxx (53.1)} &  xxx (18.1) & xxx (17.0) \\
swh & 64.75 (65.04) &  58.76 (59.81) &   \textbf{xxx (48.6)} & xxx (25.3) & xxx (18.7) \\\hline
\end{tabular}}
\caption{\label{tab:ptresult} PERs on the evaluation and development sets (latter within parentheses) before and after adaptation with PTs.  Asterisk = Eval set PER significantly lower than the PER of the MULT-L baseline in the same row.}
\end{table*}

This section demonstrates that PT adaptation improves the
generalization capability of multilingual ASR to an unseen target
language.  Adaptation to ASR-derived PTs (self-training) significantly
reduces PER, as has been previously
reported~\cite{vesely2013-semi}. PTs derived from human mismatched
crowdsourcing provide significant further PER reduction.

Table~\ref{tab:ptresult} presents phone error rates (PERs) on the
evaluation (and development) sets for four different languages. {\sc
  Mult-L} corresponds to the multilingual baseline error rates
reproduced from Table~\ref{tbl:results}.  PERs of GMM-HMM systems are
reproduced in rows 4-7; PERs of NN-HMM systems are reproduced in rows
9-12.

{\sc ST} refers to the NN-HMM multilingual baselines adapted with
automatically generated PTs (ASR self-training). Self-training was
only performed using NN systems; no self-training of GMMs was
performed.  Asterisk denotes a system whose eval-set PER is
significantly lower than the eval-set PER of the MULT-L baseline in
the same row ($p<0.05$, MAPSSWE test computed using the NIST {\tt
  sc\_stats} tool~\cite{Pallet90}).  {\bf (TODO: RUN THE TESTS, INSERT
  ASTERISKS)}

{\sc PT-adapt} corresponds to PERs from multilingual speech
recognition systems that have been adapted to PTs in the target
language. We observe substantial PER improvements using {\sc PT-adapt}
over {\sc Mult-L} across all four languages. We also find that PT
adaptation consistently outperforms the {\sc ST} systems for all four
languages. The relative reductions in PER compared to both baselines
are listed in the last two columns.  This suggests that
adaptation with PTs is providing more information than that obtained
by model self-training alone. It is also interesting that we obtain
significantly larger PER improvements with PTs for Swahili compared to
the other three languages. We conjecture this may be partly because
Swahili's orthography is based on the Roman alphabet unlike the other
three languages. Since the mismatched transcripts also used the Roman
alphabet, the PTs derived from them may more closely resemble the
native Swahili transcriptions (from which the phonetic transcriptions
are derived).

It is also useful to compare the performance of GMM-HMM systems (rows
4-7 of Table~\ref{tab:ptresult}) with the performance of NN-HMM
systems (rows 9-12).  In the {\sc MULT-L} setting, an ASR trained
using six languages is then applied to an unseen seventh language,
without adaptation; in this setting, the NN consistently outperforms
the GMM (the difference is not statistically significant, because we
have only ten minutes per language of training data, but it is
consistent across languages).  In the {\sc PT-adapt} setting, either
GMMs or NNs are adapted using PTs in the target language.  PT
adaptation improves the performance of both types of ASR, but the NN
does not improve as much as the GMM.



\section{Discussion}

Models of human neural processing systems have often been used to
inspire improvements in machine-learning systems (for example, models of
human auditory processing based on psychoacoustic studies of the
auditory system have inspired advances in speech enhancement). These
systems are often called neuromorphic, because the system is engineered
to mimic the behavior of human neural systems. In contrast to that
approach, our incorporation of EEG signals into ASR resonates with the
idea of Human Aided Computing approach used in computer
vision.~\cite{Shenoy08,Wang09} Together with our EEG work presented here,
this class of approach represents a less explored direction for design
of machine learning systems, whereby recorded neural data (rather than
neuro-inspired models) are used as a source of prior information to
improve system performance. Therefore, our work here suggests that, by
thinking about the kinds of prior information required by a (machine
learning) system, engineers and neuroscientists can work together to
design specific neuroscience experiments that leverage human abilities
and provide information that can be directly integrated into the system
to solve an engineering problem.

This paper has tentatively defined an ``under-resourced language'' to
be one that lacks transcribed speech data.  Other authors have
proposed that if a language lacks transcribed speech, ASR can be
initialized in that language by adapting a multilingual baseline
trained on other languages.  Other authors have proposed, and
Table~\ref{tab:ptresult} confirms, that significant error reductions
can be achieved using self-training: by automatically labeling speech
in the target language, and adding the self-labeled data to the
training set.  Tables~\ref{tab:ptresult} and~\ref{tab:nnpt} show that
further error rate reductions can be achieved using mismatched
crowdsourcing: by asking non-speakers of the target language to write
down what they hear, and by interpreting their nonsense orthography as
information about the phonetic content of the utterances.  The PER of
mismatched crowdsourcing (Table~\ref{tab:LPER}) is almost as high as
the PER of cross-language ASR (Table~\ref{tbl:results}), but the
information provided by mismatched crowdsourcing is superior to that
provided by self-training in the sense that it trains a better ASR.

In a sense, though, all of the results presented in this article, and
all results presented in every other article published on the subject
of under-resourced ASR, are artificial and disingenuous: ASR is
trained without deterministic transcripts, but is then tested by
comparing its output to a deterministic transcript.  In order to test
ASR in a language that truly lacks deterministic transcripts, it is
necessary to define an error metric that requires only PTs.  ASR
should be trained using the PTs in a training set (in this article, 40
minutes of speech per language), tuned using the PTs in a development
set (10 minutes per language), and tested using the PTs in a test set
(10 minutes per language).  The metric should be validated on
languages that have deterministic transcripts, by showing that the
difference between the true phone error rates of ASR systems $j$ and
$k$ ($\mbox{PER}_j-\mbox{PER}_k$) is correlated with the difference
between their probabilistic phone error rates
($\mbox{PPER}_j-\mbox{PPER}_k$).  Once this correlation has been
established for languages without deterministic transcripts, it may be
possible to use the same PPER metric to select the best-performing ASR
for a language that truly lacks deterministically transcribed speech.

(NOW THE PPER ALGORITHM AND PRELIMINARY RESULTS GO HERE)



%%%%%%%%%%%%%%%%%%%%%%%%%%%%%%%%%%
\section{Conclusions}

Transcriptions from Mismatched Crowdsourcing are very noisy.
Nevertheless, ASR adapted using Probabilistic Transcriptions beats a
multilingual ASR.

Errors in mismatched crowdsourcing are reduced using phonotactic
language models, even if those must come from text.

EEG responses can be used to estimate confusion matrices. Entropy is
lowest in native language, second lowest in a similar language, and
highest in a dissimilar language.


\section{Acknowledgments}
This work was
supported by JHU via grants from NSF (IIS), DARPA (LORELEI),
Google, Microsoft, Amazon, Mitsubishi Electric, and MERL, and by NSF
IIS 15-50145 to the University of Illinois.  Parts of
this work were previously published in~\cite{Liu15}.


\bibliography{../bib/references,../bib/refs}

\end{document}

