\subsection{Electrophysiology of Speech Perception}

The human auditory system is sensitive to within-category distinctions
in speech sounds, but such pre-categorical perceptual distinctions may
be lost in transcription tasks, where a listener must filter their
percepts through the limited number of categorical representations
available in their native language orthography.  EEG distribution
coding is a proposed new method that interprets the electrical evoked
potentials of an untrained listener (measured by
electroencephalography or EEG) as a probability distribution
over the phone set of the utterance language
(Fig.~\ref{fig:eeg_paradigm}).  A transcriber, in this scenario, listens
to speech in both his native language and an unfamiliar non-native
target language, while his EEG responses are recorded.  From his
responses to English speech, an English-language EEG phone recognizer
is trained~\cite{Liberto15}.  Misperception probabilities
$\rho(\psi|\phi)$ are then estimated: for each non-native phone
$\phi$, the classifier outputs are interpreted as an estimate of
$\rho(\psi|\phi)$.

\begin{figure}\setlength{\textfloatsep}{3mm}
\setlength{\fboxsep}{0pt}%
\setlength{\fboxrule}{0.5pt}%
\begin{center}
  \begin{tikzpicture}[
      scale=\mytikzscale,
      boxed/.style={rectangle,thick, draw=black, text=black, fill=white, 
      	rounded corners=1mm, text centered, text width=2.5cm},
      every node/.style={transform shape}      
    ]
    \node[text width=1.5in, text centered, anchor=north] (label0) at (-5,2.75) {\large EEG response to foreign phones (epoched \& averaged)};
    \node[text width=2.5in, text centered, anchor=north] (label1) at (0,2.75) {\large EEG feature classifiers trained on feature-labeled English phones};
    \node (raweeg) at (-5,0) {\fbox{\includegraphics[width=1in]{../figs/avg-eeg.pdf}}};
    \node[boxed] (f0) at (-0.4,1.1) {\large\phantom{pd}};
    \node[boxed] (f1) at (-0.3,0.9) {\large\phantom{pd}};
    \node[boxed] (f2) at (-0.2,0.7) {\large\phantom{pd}};
    \node[boxed] (f3) at (-0.1,0.5) {\large\phantom{pd}};
    \node[boxed] (f4) at (0.0,0.3) {\large\phantom{pd}};
    \node[boxed] (f5) at (0.15,0.1) {\large\phantom{p}continuant\phantom{d}};
    \node[boxed] (f6) at (0.45,-0.4) {\large\phantom{p}sonorant\phantom{d}};
    \node[boxed] (f7) at (0.75,-0.9) {\large aspirated};
    \draw[->, thick] (-3.25,0) -- (-2,0);
    \draw[->, thick] (2,0) -- (3,0);
    \node[text width=1in, text centered] (label2) at (4.5,0) {\large \baselineskip=8pt Foreign phone misperception probabilities};
  \end{tikzpicture}\\
\end{center}
\setlength{\abovecaptionskip}{0pt}
  \caption{EEG responses are recorded while the listener hears speech in
    his native language.  A bank of distinctive
    feature classifiers are trained.  The listener then hears speech in an
    unfamiliar language, and his EEG responses are classified,
    in order to estimate arc weights in the misperception FST.}
  \label{fig:eeg_paradigm}
\end{figure}
