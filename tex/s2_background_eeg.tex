\subsection{Electrophysiology of Speech Perception}

The human auditory system is sensitive to within-category distinctions
in speech sounds, but such pre-categorical perceptual distinctions may
be lost in transcription tasks, where a listener must filter their
percepts through the limited number of categorical representations
available in their native language orthography.  EEG distribution
coding is a proposed new method that interprets the electrical evoked
potentials of an untrained listener (measured by
electroencephalography or EEG) as a probability distribution
over the phone set of the utterance language
(Fig.~\ref{fig:eeg_paradigm}).  A transcriber, in this scenario, listens
to speech in both their native language and an unfamiliar non-native
target language, while their EEG responses are recorded.  From their
responses to English speech, an English-language EEG phone recognizer
is trained~\cite{Liberto15}.  Misperception probabilities
$\rho(\psi|\phi)$ are then estimated: for each non-native phone
$\phi$, the classifier outputs are interpreted as an estimate of
$\rho(\psi|\phi)$.

\begin{figure}\setlength{\textfloatsep}{3mm}
\setlength{\fboxsep}{0pt}%
\setlength{\fboxrule}{0.5pt}%
\begin{center}
  \begin{tikzpicture}[
      scale=\mytikzscale,
      boxed/.style={rectangle,thick, draw=black, text=black, fill=white, 
      	rounded corners=1mm, text centered, text width=2.5cm},
      every node/.style={transform shape}      
    ]
    \node[text width=1.5in, text centered, anchor=north] (label0) at (-5,2.75) {EEG response to foreign phones (epoched \& averaged)};
    \node[text width=2.5in, text centered, anchor=north] (label1) at (0,2.75) {EEG feature classifiers trained on feature-labeled English phones};
    \node (raweeg) at (-5,0) {\fbox{\includegraphics[width=1in]{../figs/avg-eeg.pdf}}};
    \node[boxed] (f0) at (-0.4,1.1) {\phantom{pd}};
    \node[boxed] (f1) at (-0.3,0.9) {\phantom{pd}};
    \node[boxed] (f2) at (-0.2,0.7) {\phantom{pd}};
    \node[boxed] (f3) at (-0.1,0.5) {\phantom{pd}};
    \node[boxed] (f4) at (0.0,0.3) {\phantom{pd}};
    \node[boxed] (f5) at (0.15,0.1) {\phantom{p}continuant\phantom{d}};
    \node[boxed] (f6) at (0.45,-0.4) {\phantom{p}sonorant\phantom{d}};
    \node[boxed] (f7) at (0.75,-0.9) {aspirated};
    \draw[->, thick] (-3.25,0) -- (-2,0);
    \draw[->, thick] (2,0) -- (3,0);
    \node[text width=1in, text centered] (label2) at (4.5,0) {\baselineskip=8pt Foreign phone misperception probabilities};
  \end{tikzpicture}\\
\end{center}
\setlength{\abovecaptionskip}{0pt}
  \caption{EEG responses are recorded while listeners hear speech in
    their native language.  For each listener, a bank of distinctive
    feature classifiers are trained.  Listeners then hear speech in an
    unfamiliar language, and their EEG responses are classified,
    estimating a listener-language probabilistic transcript of the
    non-native speech.}
  \label{fig:eeg_paradigm}
\end{figure}
