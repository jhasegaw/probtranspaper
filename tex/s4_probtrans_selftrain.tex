\subsection{Self Training}
\label{sec:selftraining}

The method used for semi-supervised training is a modification of the
self-training approach described in \cite{vesely2013-semi}. In this
method, a multilingual DNN-HMM speech recognizer, trained languages
other than the target language, is used to decode unlabelled audio in
the target language.  As shown in Fig.~\ref{fig:fig_hager}, decoding
results in a posterior probability $\pi(\phi_m^\ell|x_t^\ell)$ for
each frame $x_t^\ell$ of audio in the target language.

\begin{figure}
  \centerline{\includegraphics[width=5in]{../figs/fig_hager.png}}
  \caption{The self-training method of~\cite{vesely2013-semi} includes
    a labeling phase and a learning phase.  (a) Labeling phase: an ASR
    trained on other languages (here Cantonese) is used to compute
    posterior phone probabilities $\pi(\phi_t^\ell|x^\ell)$ in the
    test language (here Mandarin). (b) Learning phase: posterior phone
    probabilities are used as targets for DNN re-training.}
  \label{fig:fig_hager}
\end{figure}

We empirically found it better to use the posteriors as soft-targets
in frame cross-entropy training (Fig.~\ref{fig:fig_hager}). This is
different from the approach in \cite{vesely2013-semi}, which uses the
best path alignment as the target. Additionally, we scaled the amount
of transcribed data by 2 to create a good balance between transcribed
and untranscribed data as suggested in the original work.

The results on using this DNN are shown in Table \ref{tab:ptresult}. Although
semi-supervised training improves PER performance over multilingual DNN, it
still falls short of adaptation to probabilistic transcriptions (described in
Section \ref{sec:adaptation}). This is in spite of the untranscribed audio data
being several times larger than the probabilistic transcription data. Thus, we
show that mismatch transcripts can be more effective than ASR transcription for
training acoustic models.

