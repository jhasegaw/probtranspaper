%\subsection{Cross-Lingual Baselines}
\subsection{ASR Methods}
\label{sec:mlbaseline}

{\color{blue} Automatic speech recognition (ASR) systems were trained
  in four languages (hun=Hungarian, cmn=Mandarin, swa=Swahili,
  yue=Cantonese), using three different types of transcription.
  First, a topline {\sc monolingual} system was trained in each
  language using speech transcribed by a native speaker of that
  language.  Second, a baseline {\sc CL} (cross-lingual) system was
  trained using data from other languages, and tested in the target
  language.  Third, the experimental {\sc PT-adapt} system was created
  by adapting the cross-lingual system to probabilistic transcriptions
  in the target language.  The {\sc monolingal} topline system is
  trained using native transcripts, and converted to the phone set of
  the test language using the G2Ps described in Sec.~\ref{sec:data}.
  These resoures were not available to the {\sc CL} or {\sc PT-adapt}
  systems, which were not permitted to use any natively transcribed
  training data in the test language.}

{\color{blue} Audio data, native transcripts, and probabilistic
  transcripts are as described in Sec.~\ref{sec:data}.  The {\sc
    monolingual} topline system was trained using 40 minutes of
  training data, then stream weights and insertion penalties were
  calculated using 10 minutes of development test data.  Monolingual
  systems were trained using a maximum likelihood (ML) criterion using
  the 40 minute in-language training set: GMM parameters were
  initialized using a monophone system trained on the same 40 minutes,
  NN parameters were initialized using a restricted Boltzmann machine
  trained on five hours of unlabeled audio in the same language.  The
  {\sc CL} baseline systems were each trained using 40 minutes of
  training data in languages other than the test language.  CL systems
  were trained using ML, maximum mutual information (MMI), minimum
  phone error rate (MPE), and state-based minimum Bayes risk
  (sMBR,~\cite{Gibson06}) training criteria.  The {\sc PT-adapt}
  system was initialized using the CL system (ML training), then
  adapted to the target language using PTs based on mismatched
  crowdsourcing (these transcripts are described in detail in
  Sec.~\ref{sec:data}).  Probabilistic transcripts based on EEG were
  not used to adapt ASR, because it is not yet possible to use EEG to
  generate probabilistic transcripts on a scale sufficient for ASR
  adaptation.}

All systems were trained using the Kaldi~\cite{Kaldi2011}
toolkit. {\color{blue} Acoustic features consisted of MFCC (13
  features), stacked $\pm 3$ frames ($13\times 7=91$ features),
  reduced to 40 dimensions using LDA followed by fMLLR.  GMM-HMM
  systems directly observed this 40-dimensional vector; NN-HMM systems
  computed fMLLR+d+dd stacked $\pm 5$ frames ($40\times 3\times
  11=1320$ features/frame).  All systems used tied triphone acoustic
  models, based on a decision tree with 1200 leaves.  Each GMM-HMM
  used a library of 8000 Gaussians, shared among the 1200 leaves.
  Each NN-HMM used six hidden layers with logistic nonlinearities, and
  with 1024 nodes per hidden layer, followed by a softmax output layer
  with 1200 nodes.}
  


%The goal of building a cross-lingual system is two-fold.
%One is to define a baseline for generalizing to an unseen
%language without any labeled audio corpus.  The other
%is have the baseline serve as a starting point for
%adaptation.

%The dataset consists of 40 minutes of labeled audio for training,
%10 minutes for development, and 10 minutes for testing
%for each language.
%The orthographic transcripts are converted into
%phonetic transcripts using language-dependent G2Ps.
%Beginning with a list of the IPA symbols used in canonical descriptions
%of all seven languages,
%any symbol appearing in only one language was merged with a different symbol
%differing by only one distinctive feature; this process proceeded until 
%each remaining phone symbol is represented in at least two languages.

%Each {\color{blue} {\sc CL} baseline}
%HMM was trained with data from six languages, tuned
%(stream weight and insertion penalty)
%on the development set of the seventh language, and
%tested on the evaluation set of the seventh language.  The lexicon of
%the target language was not used during testing, but two types of
%language-dependent specialization were allowed.  In the first type of
%specialization, the universal phone set was restricted at test time to
%output only phones in the target language.  In the second type of
%specialization, a target-language phone bigram language model was
%trained using phone sequences converted from Wikipedia texts.
%text.  The texts were
%collected from Wikipedia articles linked from the main page of each
%language crawled once per day over four months.
%As an oracle experiment, we also train language dependent
%HMMs for each individual language with 40 minutes of labeled audio.


