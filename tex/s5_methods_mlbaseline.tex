\subsection{Multilingual Baselines}
\label{sec:mlbaseline}

The goal of building a multilingual system is two-fold.
One is to setup a baseline for generalizing to an unseen
language without any labeled audio corpus.  The other
is have the baseline serve as a starting point for
adaptation.

The dataset consists of 40 minutes of labeled audio for training,
10 minutes for development, 10 minutes for testing
for each language.
The orthographic transcriptions are converted into
phonemic transcriptions in the following steps.
Beginning with a list of the IPA symbols used in canonical descriptions
of all seven languages,
symbols appearing in only one language were each merged with a symbols
differing in only one distinctive feature; this process proceeded until 
each phone in the universal set is represented in at least two languages.
English words are identified and converted to phonemes with
an English G2P trained using the CMUdict~\cite{Lenco15}.
We take the canonical pronunciation of a word if the word
appears in a lexicon,
otherwise estimate the word's pronunciation using a G2P.
The Arabic dictionary is from the Qatari Arabic Corpus~\cite{Elmahdy14},
the Dutch dictionary is from CELEX v2~\cite{Baayen96},
the Hungarian dictionary was provided by BUT~\cite{Grezl14},
the Cantonese dictionary is from $I^2R$,
the Mandarin dictionary is from CALLHOME~\cite{Canavan96},
and the Urdu and Swahili G2Ps were compiled from simple
rule-based descriptions of the orthographic systems in those
two languages~\cite{Hasegawajohnson15}.

We train a standard HMM with training data from six languages,
fine-tune hyperparameters on the development set of the seventh language,
and test the model on the test set of the seventh language.
We assume that the lexicon of the target language is unknown,
but that we are allowed to restrict the universal phone set 
at test time
to output only phones in the target language.
We also assume we have access to texts of the target language,
so that we can train a phone language model on the phone
sequences converted from texts.
The texts are collected from Wikipedia articles linked
from the main page of each language crawled once per day over four months.
Results are shown in Table~\ref{tbl:results} where we compare results using
universal phone set and phone language model to those obtained using 
language-dependent phone set and phone language model.
Without a language specific phone set and phoneme language model,
it is hard for a multilingual system to generalize to
an unseen language.  This is true even if the system has seen
closely related languages such as Mandarin
when tested on Cantonese.

As an oracle experiment, we also train language dependent HMMs
for each individual language with 40 minutes of labeled audio.
Results are shown in Table~\ref{tbl:results}.  It is encouraging to see
significant improvement across all languages
when equipped with DNN even if there are only 40 minutes of
data to train the DNN.

\begin{table*}
\begin{center}
\begin{tabular}{|l|llll|}
\hline
target language & CA & HG & MD & SW \\
\hline
multilingual GMM-HMM (universal) & 79.64 (79.83) & 77.13 (77.85) & 83.28 (82.12) & 82.99 (81.86) \\
multilingual DNN-HMM (universal) & 78.62 (77.58) & 75.98 (76.44) & 81.86 (80.47) & 82.30 (81.18) \\
multilingual GMM-HMM (language specific) & 68.40 (68.35) & 68.62 (66.90) & 71.30 (68.66) & 63.04 (64.73) \\
multilingual DNN-HMM (language specific) & 66.54 (65.28) & 66.08 (66.58) & 65.77 (64.80) & 64.75 (65.04) \\
\hline
monolingual GMM-HMM & 32.77 (34.61) & 39.58 (39.77) & 32.21 (26.92) & 35.33 (46.51) \\
monolingual DNN-HMM & 27.67 (28.88) & 35.87 (36.58) & 27.80 (23.96) & 34.98 (41.47) \\
\hline
\end{tabular}
\caption{\label{tbl:results} PERs of unadapted
multilingual systems on the evaluation sets along with monolingual systems.
PERs on the development sets are in parentheses.}
\end{center}
\end{table*}

From the comparison of different baseline systems, we can reach the
following conclusions.  First, the standard speech pipeline is able to
train speech recognizers using SBS data.  Even with only 40 minutes of
training data, a DNN is able to outperform a GMM-HMM.  Second,
however, the standard speech pipeline generalizes well to languages
that were seen in the training corpus, but performs poorly on unseen
languages.  Using a language-specific phonotactic language model gives
significant improvement over the language-independent phonotactic
model, but nevertheless significantly underperforms any system that
has seen the test language during training.  

