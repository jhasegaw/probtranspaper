%\subsection{MAP Adaptation to Probabilistic Transcripts}
\label{sec:ptadapt}

{\color{blue} The {\bf PT-adapt} system was
  adapted using MAP adaptation (Sec.~\ref{sec:adaptation})
  computed over weighted finite state transducers in }
%The baseline and the adapted models were implemented using
Kaldi~\cite{Kaldi2011}. In order to efficiently carry out the required
operations on the cascade $H\circ C\circ PT\circ G$, $PT$ was defined
as $\mathrm{proj}_{\mathrm{input}} (\widehat{PT})$, where
$\widehat{PT}$ is a wFST mapping phone sequences to English letter
sequences (Eq.~\ref{eq:PT}), and $\mathrm{proj}_{\mathrm{input}}$
refers to projecting onto the input labels. For the purposes of
computational efficiency, the cascade for $\widehat{PT}$ includes an
additional wFST restricting the number of consecutive deletions of
phones and insertions of letters (to a maximum of 3). Two additional
disambiguation symbols~\cite{mohri2008speech} were used to determinize
these insertions and deletions in $\widehat{PT}$. MAP adaptation for
the acoustic model was carried out for a number of iterations (12 for
yue \& cmn, 14 for hun \& swh, with a re-alignment stage in iteration
10).
