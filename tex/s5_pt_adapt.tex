\subsection{MAP Adaptation to Probabilistic Transcripts}
\label{sec:ptadapt}

The baseline and the adapted models were implemented using
Kaldi~\cite{Kaldi2011}. In order to efficiently carry out the required
operations on the cascade $\mathbf{AM}\circ\mathbf{HC}\mathbf{PT}$, we
carefully designed $\mathbf{PT}$ as an acceptor defined as
$\mathrm{proj}_{\mathrm{input}} (\widehat{PT})$, where $\widehat{PT}$
is a WFST mapping phone sequences to English letter sequences obtained
as a cascade of WFSTs modeling the distributions shown in
Equation~\ref{eq:PT}, and $\mathrm{proj}_{\mathrm{input}}$ refers to
projecting onto the input labels. For the purposes of computational
efficiency, the cascade for $\widehat{PT}$ includes an additional WFST
restricting the number of consecutive deletions of phones and
insertions of letters (to a maximum of 3 in our experiments). We use
two additional disambiguation symbols~\cite{mohri2008speech}, apart
from the ones used in typical Kaldi recipes, to determinize these
insertions and deletions in $\widehat{PT}$. MAP adaptation for the
acoustic model was carried out for a number of iterations (12 for yue
\& cmn, 14 for hun \& swh, with a re-alignment stage in iteration 10).
