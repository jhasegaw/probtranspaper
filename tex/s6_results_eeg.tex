%\subsection{Misperception Transducer Trained Using EEG}
\label{ssec:eeg}

\newcommand{\specialcell}[2][c]{%
  \begin{tabular}[#1]{@{}c@{}}#2\end{tabular}}

Epoched and feature-coded EEG data {\em for the English syllables
only} were used to train a support vector machine classifier for each
distinctive feature.  The classifiers were then used (without
re-training) to classify the EEG responses to the Dutch and Hindi
syllables.  Fig.~\ref{fig:eeg_svm_eers} shows equal error rates of
these classifiers when applied to the three languages.  {\color{blue}
EER of the classifier when applied to English phones is comparable to
those reported in~\cite{Liberto15}, the only prior work to attempt a
recognition of speech phonemes from EEG of the listener.}

\begin{figure}
  \centerline{\includegraphics[width=\columnwidth]{../figs/eer-barplot/eer-barplot.pdf}}
  \vspace*{-0.3cm}
  \caption{Classifiers were trained to observe EEG signals, and to
    classify the distinctive features of the phone being heard.  Equal
    error rates are shown for English (the language used in training;
    train and test data did not overlap), Dutch, and Hindi.  Dashed
    line shows chance=50\%.}
  \label{fig:eeg_svm_eers}
\end{figure}

Eq.~(\ref{eq:dfdist}) defines a log-linear model of $\rho(\psi|\phi)$,
the probability that a non-English phoneme $\phi$ will be perceived as
English phoneme $\psi$.  Denote by $\rho_U(\psi|\phi)$ the model of
Eq.~(\ref{eq:dfdist}) with uniform binary weights for all distinctive
features. Denote by $\rho_{EEG}(\psi|\phi)$ the same model, but with
weights $w_k$ derived from EEG measurements (Eq.~(\ref{eq:eegdist})).
Fig.~\ref{fig:eeg_confusions} shows these two confusion matrices:
$\rho_U(\psi|\phi)$ on the left, $\rho_{EEG}(\psi|\phi)$ on the
right. The entropy of the binary weighting, $\rho_U(\psi|\phi)$, is
too low: when a Dutch phoneme $\phi$ has a nearest-neighbor
$\psi^*(\phi)$ in English, then few other phonemes are considered to
be possible confusions.  $\rho_{EEG}(\psi|\phi)$ has a very different
problem: since distinctive feature classifiers have been trained for
only a small set of distinctive features, there are large groups of
phonemes whose confusion probabilities can not be distinguished
(giving the figure its block-matrix structure).  The faults of both
models can be ameliorated by averaging them in some way, e.g., by
computing the linear interpolation
$\rho_I(\psi|\phi)=(1-\alpha)\rho_U(\psi|\phi)+\alpha\rho_{EEG}(\psi|\phi)$
for some constant $0\le\alpha\le 1$.

\begin{figure}
  \centerline{
    \includegraphics[width=\columnwidth]{../figs/confusion-matrix/confusion-matrices.pdf}
  }
    \vspace*{-0.3cm}
  \caption{Phone confusion probabilities between English and Dutch
    phones using models in which the negative log
    probability is proportional to unweighted or weighted
    distance between the corresponding
    distinctive feature vectors.  Left: unweighted.
    Right: feature weights equal negative log confusion
    probability of EEG signal classifiers.}
  \label{fig:eeg_confusions}
\end{figure}

In order to evaluate the effectiveness of the EEG-induced
misperception transducer we looked at the LPER of mismatched
crowdsourcing for Dutch when performed using 1)~a multilingual
misperception model $\rho(\lambda|\phi)$ (the machine translation
model described in Sec.~\ref{sec:MC}), 2)~feature-based misperception
transducer computed using binary weighting, $\rho_U(\psi|\phi)$, or 3)
EEG-induced transducer combined with the feature-based transducer,
$\rho_I(\psi|\phi)$.  {\color{blue} Both method (2) and method (3)
required the use of a G2P in order to compute $\rho(\lambda|\psi)$:
the Dutch G2P was estimated using the CELEX database, while the Hindi
G2P was estimated using the zero-resource knowledge-based method
described in Sec.~\ref{sec:trainwithlm}.  The constant $\alpha=0.29$ was
chosen as the average of the values selected by all folds in a
leave-one-out cross-validation.}  LPER of the multilingual model was
70.43\% {\color{blue} (as shown in Table~\ref{fig:pt_decode_per})}, of
the feature-based model, 69.44\%, and of the EEG-interpolated model,
68.61\%.

%As shown in Table~\ref{tbl:eegresults}, 
%phone error rates were improved $1.5\%$ and $2.5\%$ relative to 
%mismatched crowdsourcing by using the feature-based and combined 
%misperception transducers (respectively).
%\begin{table*}
%\begin{center}
%\begin{tabular}{|c|c|c|c|}
%  \hline
%  & \specialcell{multilingual} & feature-based & \specialcell{EEG-induced\\+feature-based} \\
%\hline
%LPER & 70.43 & 69.44 & 68.61 \\
%\hline
%\end{tabular}
%\vspace*{1mm}
%\caption{\label{tbl:eegresults} Comparison of LPERs for probabilistic transcript of the Dutch evaluation set using different schemes to compute misperception G2Ps.}
%\end{center}
%\end{table*}


