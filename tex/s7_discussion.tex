\section{Discussion}

Models of human neural processing systems have often been used to
inspire improvements in machine-learning systems (for example, models of
human auditory processing based on psychoacoustic studies of the
auditory system have inspired advances in speech enhancement). These
systems are often called neuromorphic, because the system is engineered
to mimic the behavior of human neural systems. In contrast to that
approach, our incorporation of EEG signals into ASR resonates with the
idea of Human Aided Computing approach used in computer
vision.~\cite{Shenoy08,Wang09} Together with our EEG work presented here,
this class of approach represents a less explored direction for design
of machine learning systems, whereby recorded neural data (rather than
neuro-inspired models) are used as a source of prior information to
improve system performance. Therefore, our work here suggests that, by
thinking about the kinds of prior information required by a (machine
learning) system, engineers and neuroscientists can work together to
design specific neuroscience experiments that leverage human abilities
and provide information that can be directly integrated into the system
to solve an engineering problem.

((THIS WOULD PROBABLY BE A GOOD PLACE TO INTRODUCE PROBABILISTIC PHONE ERROR RATES FOR LANGUAGES WITH NO TEST DATA!!))
